\label{detection}
Detection is the Achilles' heel of adaptive traffic control systems. An otherwise perfect prediction algorithm will provide flawed information to the optimization procedure, if detectors are misplaced or there are too few of them. Even with a dense detector network lag issues may cause problems in the prediction and timing.

In Copenhagen upstream detectors are connected to the approaching signal controller which relay data to an area-specific hub that is monitored by a central computer. Most of the signals are transmitted via cables, which are fragile to heavy traffic, frosty weather and are often severed by entrepreneurs, and so wireless, IP based alternatives are being explored.

The most common type of detector in the areas supervised by Centre for Traffic and the Danish Road Authority is the induction loop. It is set into the tarmack and gives an indication whenever a metallic object (vehicle) passes. To detect buses extra large loops, which can only be triggered by large vehicles, are used. 

These authorities also use video detection and radar. Video is relatively cheap and can be installed without obstructing the traffic. Radar functions a bit like video but is more reliable under bad weather conditions and gives depth information.

Video is the most promising detection technology due to its rapid development and the possibility to perform post-processing on a computer to obtain additional informations besides vehicle detection. For instance, video images are being used by both Centre for Traffic Copenhagen and by Danish Road Directorate on the ring-roads to measure travel speeds (a MOE). This is done by reading parts of license plates at point A and measuring the time for it to reach some other point, which is filmed.