\label{sec:signal_types}

At present there are three major types of traffic signal systems
called {\em pretimed signals, actuated signals} and {\em adaptive
signals}. They are the result of incremental improvements described in
chronological order below:

Pretimed signals uses static plans for phase sequences, cycle time and
green splits according to the day of time.
\label{insec:pretimed}
They are based on the assumption that demand is fairly stable within
certain divisions of time eg. morning, midday and evening or
workday/weekend. For instance, in the morning (7am to 8.30am) and
afternoon (15.30pm to 17.00pm) the traffic is usually heavier than
during the day or night due to the commuters.

Traffic may prove to be more dynamic, though, and therefore the
utilization of these signals must be monitored on a regular basis so
proper adjustments can be made.

Actuated signals function like pretimed signals but with the ability
to \textit{lengthen} the green period with a certain amount, if
additional vehicles are observed.
\label{insec:actuated}
To achieve this the signal needs detector input about the demand it
faces for each phase.  A major disadvantage due to the autonomous
operation is that it becomes impossible to setup green waves since
signals start their cycle at arbitrary offsets and are unlikely to
share a common cycle time.

Adaptive signals is a network of signals which attempts to
optimize some MOE in an manner which is at least as intelligent as for
actuated signals.

The key to adaptive signals is reliable detection and short term
prediction of traffic. Adaptive signals must be able to respond to the
dynamic aspects of traffic, which are not captured in the design of
pretimed signal plans. Adaptive signals use historical input and
current detector input to make short term predictions for what is
going to happen eg. within the next minute, next 10 minutes and so
forth. For this reason, and as stated in \cite{1}, adaptive systems
are not truly adaptive because they rely on these short term
predictions and thus will always lag behind the actual traffic. For
short term prediction methods from time series analysis eg. ARIMA
models (see eg. \cite{shortpredict}) can be used to get more accurate
predictions.

An isolated adaptive signalized intersection has advantages over
actuated signals because it can skip a phase to give priority to a
bus, for instance. The main attraction with adaptive signals is when
they can be set to work together. A good system will naturally cause
green waves to appear and move the direction of the green waves along
with changes in flow.
