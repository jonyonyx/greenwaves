\section{Introduction}

Road traffic is an essential part of modern society and has put a high
demand on road networks. Figures from a recent study from the Danish
Ministry of Transport \cite{47} show traffic has increased by 50\%
since the 80's and that cars and busses are responsible for more than
90\% of all transport of people in Denmark (totalling 68 million
kilometres).

Traffic network congestion causes delays which add substantial costs
to society and businesses on a daily basis and also increase emissions
and the risk of accidents. The study mentioned above reports that in
2002 people where spending 100,000 hours in total in queues in the
Greater Copenhagen road infrastructure, this corresponds to an
economic loss of more than 750 million Euros.

To alleviate congestion, public transport can be improved or the
infrastructure can be expanded. In urban areas, the latter is often
impossible due to residential areas adjacent to the existing roads.  A
more subtle way to improve the network performance is to make better
use of the existing roads, which can be achieved in part by proper
setting of traffic signal parameters.

It is estimated that the proper use of intelligent traffic systems
including intelligent traffic signals, could increase the capacity of the
road network in the Greater Copenhagen area by 5 to 10\%.

The literature has many suggestions for the intelligent setting of
traffic signals, ranging from purely statistically based methods
developed in the early 60's, over actuated traffic signals, to highly
adaptive and cooperative methods, which can be realized using actual
flow information supplied by traffic detectors. This paper gives a
survey of the literature with special emphasis on adaptive methods
which attempt to coordinate traffic signals in larger networks so as
to optimize some network-wide performance index such as number of
stops or delays.

The paper is structured as follows. Initially the most important
definitions are presented in section \ref{sec:vocabulary}, followed by
section \ref{sec:theproblem} that describes the measurement of
performance for an intelligent traffic signal, and section
\ref{sec:model} defining the mathematical model to describe the entire
setup. The next three sections review the different traffic signal
types and the underlying optimization models. Section
\ref{sec:adaptive_cooperation} describes the adaptive and cooperative
systems that define the current state-of-the-art. Finally in section
\ref{sec:conclusion} we summarize the findings and make with
recommendations for future work.
