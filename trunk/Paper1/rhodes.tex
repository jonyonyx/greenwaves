\label{sec:rhodes}

RHODES' approach to traffic signal optimization is a hierarchical one
with 3 layers of detail, see Figure \ref{fig:rhodes_hierarchi}.

The macroscopic layer performs \textit{dynamic network loading}, which
involves observing changes in the aggregated flow data of the entire
network due to variations in the OD matrices. This layer supplies
estimates of link flows to the middle level in rough numbers
eg. vehicles per hour.

The mesoscopic middle layer considers sectors of the network eg. an
arterial. This \textit{network flow control} layer works in the detail
level of platoons and average speeds. Green time is allocated to
phases to accommodate the movements of the platoons and so
coordination of intersections is done at this level.

At the lowest level is \textit{intersection control} where vehicles
are handled individually (a microscopic layer). Here the green times
and phase ordering suggested by the middle layer are fine tuned.

\begin{center}
{\bf Figure 2 in here}
\end{center}

An adaptive traffic control system must operate quickly in order to
adapt signals to traffic in real-time. The RHODES platform has good
decomposition opportunities and is pluggable ie. the upper level is a
black box feeding the lower level with predictions and
optimizations. At the time the article was written, the top level of
RHODES had not received much development and thus only the middle and
lower level are described herein.

\begin{center}
{\bf Figure 3 in here}
\end{center}

\subsubsection*{Prediction}

The PREDICT method \citet{PREDICT} by the co-author, Larry Head, is
used to make predictions for individual vehicles. PREDICT is built for
network prediction and relies on the fact that incoming flow to an
intersection originates from adjacent intersections. This concept can
be explained from Figure \ref{fig:rhodes_predict} where traffic
detected at $d_a$ is the sum of right-turning traffic at detector
$d_r$, left turning traffic at $d_l$ and through-going traffic at
$d_t$.

Thus, given flow estimates for the links facing intersection B, and
turning probabilities for each link, an estimate can be given for the
inflow to intersection A from east. On the link between the two
intersections there will be traffic entering and exiting the system,
but these contributions - and losses - to the traffic, which can be
measured at $d_a$, are expected to be very small.

Prediction of arrival times of the vehicles which have passed detectors
$d_{\lbrace r,t,l \rbrace}$ depend on the current phase at
intersection B and queue conditions. Mirchandani and Head have
identified four cases, which cover arrivals at an intersection, which
are summarized in table \ref{tbl:delaycases}.

\begin{center}
{\bf Table 3 in here}
\end{center}

In the cases involving queue there is, of course, a possibility that
the vehicle will not be able to cross the intersection before several
green phases have occurred. This is likely to happen under high
congestion when intersections are placed close together.

At the mesoscopic level, network flow control, the APRES-NET
prediction method, is used. It is based on simulation and has
similarities to PREDICT, though it works in the detail level of
platoons and encompasses several intersections, not just the upstream
ones but also those upstream of the upstream intersections and so
on. Since the 2nd level must deliver complete suggestions for timing
plans for each intersection (to be fine tuned by intersection control)
it must run quickly. Performance is dependent on the number of
intersections in the monitored sector, and sector sizes can thus be
scaled to match the speed requirements depending on hardware.

The prediction horizon for network flow control is 200-300
seconds. Cycle times for simple intersections with just a couple of
phases vary between 60 and 150 seconds so this horizon is adequate to
predict and respond to most types of fluctuations by performing phase
skipping, phase reordering and assignment of phase durations. The
intersection level control operates with a prediction horizon of 20-40
seconds and thus can only make decisions on whether to lengthen or
decrease the green time of phases within that horizon.

\subsubsection*{Optimization}
As in the dynamic model of section \ref{sec:dynamicmodel}, timing
plans are described by phase ordering and duration independent of
cycle time, splits and offsets.  Optimization is performed on each
level using prediction results for that level.

At the network flow control level the REALBAND algorithm forms
progression bands (ie. \textit{green waves}) for platoons traversing
the sector based on the predictions from APRES-NET. This is done by
finding \textit{conflicts} between platoons, which will request access
for conflicting phases at the same time. In this way a conflict can be
regarded as the denial of green to a platoon due to the passage of
another platoon. A decision tree within the optimization horizon of
200-300 seconds is build and explored to find the configuration with
the fewest conflicts. This results in a set of phase orderings and
green times for each intersection.

At the lower level a dynamic programming approach, COP, takes the
results from REALBAND and distributes green time for some horizon to
the phases received from the above level. The phases and their
ordering must be respected so as to not introduce conflicts which
have been resolved by REALBAND. For the same reason there are
restrictions for the maximum change in either direction of the given
green times, but COP is allowed to use its more detailed predictions
to perform the above fine-tuning of green times.

\subsubsection*{Evaluation}
RHODES has been implemented and evaluated in CORSIM as part of the
evaluation for the Federal Highway Administration (FHWA) inclusion in
RT-TRACS. RT-TRACS is an effort to choose and standardize a peak
performance traffic signal optimization system for American traffic
networks.

The simulation is done for an arterial of 9 intersections with steady
increase and then decrease of traffic over a 2-hour period. This is a
FHWA test case and the baseline traffic control system is
semi-actuated control based on the results of offline tools including
TRANSYT and PASSER, which represent the best-can-do from an offline
approach and can be considered a hard competitor.

Testing shows that RHODES is more capable of exploiting the capacity
of the arterial. As long as there is no congestion the throughput will
match the demand and in the comparison RHODES can simply take more
load before experiencing congestion.

Real adaptive systems should excel in the case of low demand, since
the overcapacity will then allow RHODES to, roughly said, cater for
each vehicle. The effect, compared to the semi-actuated control, is
convincing with 50\% reduction in delays for low demand and 30\%
reduction for high demand. This effect is expected to disappear when
demand reaches the capacity of the arterial in which case, for both
systems in the comparison, throughput can only be improved by
increasing the green time along the arterial and maintaining proper
coordination.

The simulation was run multiple times and for both throughput and
delay it is clear that RHODES is more consistent than semi-actuated
control and offers less variability from run to run.
