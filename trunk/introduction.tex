\section{Introduction}
Road traffic has become an essential part of modern society and has put a high demand on road networks supplied by municipalities. 

Traffic network congestion causes delays which costs the society and businesses substantial amounts on a daily basis and also increase emmisions and the risk of accidents.

To alleviate congestion public transport can be improved or the infrastructure can be expanded. In urban areas, the latter is often infeasible due to residential areas adjacent to the existing roads. 
A more subtle way to improve the network performance is to make better use of the existing roads. This is where traffic signals comes into the picture. 

Traffic signals are employed in urban areas in order to control traffic flows and avoid collisions by fairly distributing the right-of-way for adjoining road segments and to ensure a smooth flow of traffic.

If not set appropriately they may seriously hinder the traffic flow through the road network and so great care should be taken when choosing the parameters for an intersection. 

The litterature has many suggestions for the intelligent setting of traffic signals, ranging from purely statistically based methods developed in the early 60's over actuated traffic signals to highly adaptive and cooperative methods, which can be realized using actual flow information coming from detectors. 

This report gives a survey of the litterature with special emphasis on adaptive methods, which attempts to coordinate traffic signals in larger networks so as to optimize some network-wide performance index. 

The report is structured as follows. In section \ref{vocabulary} the basic vocabulary of traffic control is summarized. Then the objective and scope of the report is stated in section \ref{scope}. 
The optimization problem and the traffic lights being optimized are described in sections \ref{theproblem} and \ref{signal_types}.
Section \ref{history} gives an outline of some important research results in the field. 
The classic traffic optimization paradigm for offline optimization is described in section \ref{offline}.
Then in section \ref{adaptive_cooperation} the attention is turned to adaptive systems and three systems for intersection, arterial and network optimization are described. 

Simulation is an essential component for establishing confidence in adaptive traffic control systems and is also widely to assess the performance of a proposed signal plan. In the final section \ref{evaluation} simulation for these purposes is discussed.