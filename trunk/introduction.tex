\section{Introduction}
Ground based vehicular traffic has evolved rapidly over the last century. This has put a high demand on the road networks supplied by the municipalities and in Denmark, for instance, this is estimated to give rise to extensive economic loss due to waisted time \cite{do2006kap3}. 

The main problem faced due to the the increasing demand is the delays caused by traffic network congestion. Delay costs the society and businesses substantial amounts on a daily basis and also increase emmisions and the risk of accidents.

The obvious way to reduce congestion is to add more roads or add extra lanes to existing roads. But especially in urban areas, this is often infeasible due to residential areas adjacent to the existing roads. 
A more intricate way to reduce delays is to make better use of the existing roads. This is where traffic signals comes into the picture. 

Traffic signals are mostly employed in urban areas in order to control traffic flows and avoid collisions by fairly distributing the right-of-way for adjoining road segments.

But if not set appropriately they may also seriously hinder the traffic flow through the road network. This happens, for instance, when a green light phase is stopped before a platoon of vehicles have passed or a red light is given to a road segment, which has large amounts of cars waiting to pass.

The litterature has many suggestions for the intelligent setting of traffic signals, ranging from purely statistically based methods in the early 60's over actuated traffic signals to highly adaptive and cooperative methods, which can be realized using modern sensor technology. 

This report gives a survey of the litterature with special emphasis on adaptive methods, which attempts to coordinate traffic signals in larger networks so as to optimize some network-wide performance index.

The report is structure as follows. In section \ref{vocabulary} the basic vocabulary of traffic control is summarized. Section \ref{history} gives an outline of the research done in the traffic signal optimization. Then in section \ref{adaptive_cooperation} the attention is turned to adaptive systems with high degrees of cooperation.
Finally, in section \ref{conclusion}, are recommendations on methods and approaches for future work.