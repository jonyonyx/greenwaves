\label{model}
The problem is now formulated as the minimization of a selected performance index (PI) in terms of a set of signal settings, $\Psi$, and a network, which is dependent on the signal settings.

$$\min PI \left(  \Psi, N\left( \Psi\right)  \right)  $$

In a general, discrete time, model time each time steps are denoted by $t$. The network consists of $N \geq 1$ signal controlled intersections indexed by $n$.

In the model each signal is designated a phase for each time unit. Thus the concept of a cycle becomes virtual as they are no longer mandatory for calculating eg. the length of phases given the green splits.

Without a common cycle time - or individual cycle time, even - the offset parameter also disappears. However they can be made to exist virtually, in terms of a virtual cycle, and can thus be manipulated to excert the same behaviour. 
The main problem is during initialization when the system has just started. In this case it is possible synchronize intersections by delaying startup for those that would otherwise have a positive offset and vice versa for negative offset intersections. The same strategy can be used when increasing or decreasing the (virtual) cycle time for the arterial.

Phase sequences and green splits are unified by the specification of the phase in a timeslot, $t$, referred to as $p_n(t)$. For each intersection there will be a fixed number of phases, $P_n$, which are free from right-of-way conflicts. For a simple cross intersection with left-side driven vehicles and left turning priority this number is 4:

\begin{description}
\item[Phase 1 and 2:] straight and right turning flows in N-S and E-W directions
\item[Phase 3 and 4:] left turns in N-S and E-W directions
\end{description}

Thus a specific phase $p = p_n(t) \in \lbrace 1,...,P_n \rbrace$ is selected for each timeslot. With this definition the green splits are implicit in the phase sequence. 

Now $\Psi = \lbrace \textbf{p} \rbrace $ and we can perform an optimization.

Satisfaction of minimum and maximum green times is the most common constraint, usually defined within a cycle. In this model the following equation must be satisfied:

$$
\forall p,n: \; T = \lbrace t: p_n(t) = p \rbrace \; st. \; cons(T) \wedge T_{min,p,n} \leq |T| \leq T_{max,p,n} 
$$
Thus $T$ is a set of points in time and the $cons$ function is defined by:
$$
cons(T) = 
\begin{cases}
true & if \;  \displaystyle\sum_{t \in T}{t} = � \cdot (\max T + \min T) \cdot (\max T - \min T + 1) \\
false & otherwise
\end{cases}
$$

That is, there must be a consecutive series of time slots (tested by using the Gaussian formula) in which phase $p$ is run.