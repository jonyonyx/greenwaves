\label{history}
Some of the first public research into the optimization of traffic signals was published by Wardrop \cite{Wardrop} and Webster \cite{Webster}. Wardrop determines the Stochastic User Equilibrium essential for network assignment and the prediction of demand. 

\subsection{Delay and Queue Models}
Proper estimation of delays at an intersection is important in the design of signal plans.

Webster gives an approximate formula for average delay for isolated intersection with a fixed timing plan under steady state and undersaturated conditions ie. the flow cannot be dynamic and the inflow. 
He uses his own result to give expressions for the optimum cycle- and green times. 
Websters results have been widely used in the litterature eg. in \cite{1} to generate initial solutions of cycle length and green splits for a metaheuristic search and in \cite{30} to calculate cycle lengths.

The problem with Websters delay formula, along with other delay formulas developed from queuing theory is the assumption of steady state conditions. As the load on the intersection increases it will take longer time to reach stochastic equilibrium \cite{traffictheory}. During the stabilization period the signal settings must remain fixed, which it will never be in adaptive systems. These types of models are still used see eg. \cite{38} where Mirchandani and Zou develops a FIFO single-server queueing model with Poisson arrivals.

Time series analysis and other \textit{moving average} techniques can be used to relax these assumptions. In \cite{shortpredict} an ARIMA process is used to make short term (1 minute) predictions of arrivals. The RHODES system \cite{44} makes short term predictions on multiple levels of resolution (single-vehicle and platoons). In \cite{1}                                                                                          a Poisson process is used to calculate the interarrival time, with the parameter being the mean arrival rate thus anticipating the dynamic nature of traffic.

\subsection{Traffic Assignment}
Traffic signals are set to accommodate the flow of traffic thus it is vitally important to \textit{know the flow} in advance. Wardrop contributed with two principles concerning traffic flow:

\begin{enumerate}
\item \textbf{User Equilibrium:} (UE) each trafficant has minimized his own travel time (greedy route choice)
\item \textbf{System Equilibrium:} (SE) the average travel time is minimized (coordinated route choice)
\end{enumerate}

Given the choice a user will select the route which he \textit{perceives} to be the best. This route does not necessarily correspond with the actual shortest route, which has led to the Stochastic User Equilibrium (SUE) variation where this error is modelled by a stochastic element. 

The SUE is the most realistic model since 1) trafficants may not have perfect route information and may also choose a longer route on purpose (for the scenery, eg.) and 2) trafficants, at present, cannot communicate in order to obtain SE. In addition the SE entails that some trafficants may not have an optimal route, this kind of sacrifice will be difficult to accept.

In order to perform a traffic assignment for a network flow data is obtained from traffic counts or detector input over a time period. Auxilliary informations such as turning directions and vehicle types are collected as well, if possible. 

This data is fitted by a stochastic process and the optimization is made on the assumption that the process can describe the actual arrivals.

Thus there are two problems 1) determining demands and 2) optimizing the traffic signals accordingly.

It is important to realize that changing the traffic signal settings will cause changes to the flow and that changing the flow should cause the signals settings to be recalculated. This has given rise to a number of papers considering the two problems at the same time in so-called bilevel formulations eg. 
\begin{itemize}
\item \textbf{\cite{mc}:} iterative procedure which optimizes traffic signals and then solves the traffic assignment problem untill mutual consistency (convergence) is acheived
\item \textbf{\cite{34}:} gradient projection method for finding local optima combined with a global heuristic search
\item And in \textbf{\cite{2}} using a genetic algorithm approach.
\end{itemize}

A feature of adaptive systems is that they are less dependent on a large database of historical flow data since they use online data input from detectors, which they are, in turn, more dependent on. Thereby they do not have to consider the problem of mutual equilibrium between signal settings and user equilibrium.

\subsection{Detection and Signal Control}
Detection is the Achilles' heel of adaptive traffic control systems. An otherwise perfect prediction algorithm will provide flawed information to the optimization procedure, if sensors are misplaced or there are too few of them. Even with a dense sensor network lag issues may cause problems in the prediction and timing.

In Copenhagen upstream sensors are connected to the approaching signal controller which relay data to an area-specific hub, which is monitored by a central computer. Most of the signals are transmitted via cables, which are often severed by entrepreneurs so wireless, IP based alternatives are being explored.

The most common type of sensor in the area of Copenhagen is the induction loop. It is set into the tarmack and gives an indication whenever a metallic object (vehicle) passes. To detect buses extra large loops, which can only be triggered by large vehicles, are used. 

The Centre for Traffic also use video detection and radar. Video is relatively cheap and can be installed without obstructing the traffic. Radar functions a bit like video but is more reliable under bad weather conditions and gives depth information.
Video is the most promising detection technology due to its rapid development and the possibility to perform post-processing on a computer to obtain additional informations besides vehicle detection. For instance, video images are being used in Copenhagen to measure travel speeds (a MOE). This is done by reading parts of license plates at point A and measuring the time for it to reach some other point, which is filmed.