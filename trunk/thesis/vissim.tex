\subsection*{Vissim}
\label{vissim}
Vissim provides a microscopic simulation of traffic in a network. In this context microscopic entails that each vehicle is individually modelled and, in a sense, Vissim takes the view of the users of the network.

The main feat of Vissim is the visual representation of vehicle movements and inspection of signal controller states. Furthermore Vissim allows a multitude of information to be extracted in addition to the ones listed below. This, and a theoretically well foundedness, is presumably why Vissim has become so widespread in Denmark.

\begin{enumerate}
\item Delay: the difference between the best-case travel time ie. no intersections and travel at free flow speeds (no other road users). The delay from point A to B when there are no intermediate intersections is an indication of the congestion
\item Travel time is measured as the raw time spent by road users getting from A to B in a network. Travel time sections can be placed anywhere and, like delay, can be filtered on vehicle type eg. car, truck and bus
\item Average queue lengths are correlated with delays especially when measuring delays for vehicles crossing an intersection
\end{enumerate}

Below I will briefly introduce the Vissim elements that have required the most attention in this project. In section \ref{modelling} I will go into the practical details of link inputs and route choices.

\subsubsection*{Network Elements}

Vissim is highly flexible with regards to network layout description and intersection turning movements can be fine-tuned almost to perfection, given enough time. The network is designed using a GUI interface where links are joined using connectors.

Links are directed sections of road with one or more lanes. All lanes have the same direction as the link - the opposite direction is modelled using a parallel link.

Connectors are mostly relevant in intersections where they connect exactly two links, however the lanes connected within the links can be chosen freely.

\subsubsection*{Traffic input}
Vissim is capable of dynamic traffic assignment (DTA) as well as static input. DTA is a method by \cite{Wardrop} in which the traffic flow adapts to the capacity of the network due to a \textit{stochastic user equilibrium} (SUE) ie. the road user response of traffic conditions model their choice of route. 

Vissim can perform this form of traffic input given zones, which are confined and non-overlapping areas of the network, and a matrix describing the relative traffic from each zone to each other zone.

In this project DTA was abandoned on recommendation from DTU Transport professor Otto Anker Nielsen due to the dynamic nature of the traffic signals since, in his experience, odd phenomenon may occur such as random gridlocks, when adjacent links are fully saturated.

Instead static input was chosen. In static input traffic enters on specific links and traverse the network until an exit link has been reached. Static input is used in combination with routing decisions, which are usually placed on input such that each vehicle entering will routed through the network according to the available traffic data.

\subsubsection*{Routing decisions}
\label{routingdecisions}
Routing decisions are made by vehicles when they pass over a routing decision point in the Vissim network. Selecting routes for vehicles can be done either by placing a routing decision point on an input link and designate a number of exit link options (end-to-end routes) or by placing routing decisions on each approach and route vehicles through eg. an intersection (local routes).

For the first option there exist many routes even for small networks when routes are symmetric (albeit using different links). Furthermore the number of routes will grow exponentially as the number of exit links (intersections) increase and the problem of mapping traffic count data onto end-to-end routes is not easy to solve. On the upside end-to-end routes are more correct to use since road users will most likely know how they wish to traverse the entire artery in advance and routes ending in an exit link is certain to cause no vehicles to cycle about indefinately in the network.

The second option requires a route per traffic stream in each intersection. Here a traffic stream is a pair of orientational markers indicating from where the route origins and where it ends eg. north-to-south (a through-going traffic stream) or north-to-west (a left-turning stream). This option is easy to implement given traffic count data since they immediately give the proportions of vehicles making each turn (ie. taking each route) in an approach. A vehicle will pass a number of routing decisions, obtaining new routes (making turning movements) until an exit link is reached.

It is possible to implement the first option by solving a linear program using the ordinary traffic counts as input data. The complexity of this solution caused me to adopt the second option instead and in section \ref{routefractions} I describe how I did this. Note that the problem of indefinite cycling the network cannot occur since I am regarding an artery only in this project and thus it is not possible to choose a cyclic route (there are no u-turns).
