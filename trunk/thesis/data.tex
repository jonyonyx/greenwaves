\section{Traffic Analysis}
\label{data}
For the two areas of interest, Glostrup and Herlev, TTS supplied data from detectors for a period of a couple of weeks in november 2007. 

This period is generally accepted as the beginning of the winter season and as such the traffic is expected to be high.

This is a supplement to the traffic counts, which were given by DRD. Since the detector data is anonymous with respect to vehicles types, for links in the ends of the arterial (Herlev Sygehus from north and Roskildevej from east, west and south) the data from the corresponding detectors was used in the project to scale the input sizes of the traffic counts.

The format of the data is:

\begin{table}[!ht]
\begin{center}
\begin{tabular}{c|c|c|c|c}
\textbf{Date} & \textbf{Time} & \textbf{D1} & \textbf{...} & \textbf{DN} \\ \hline
13-11-2007 & 08:42:56 & 39 & ... & 35 \\
13-11-2007 & 08:44:26 & 38 & ...  & 28 \\
\end{tabular}
\end{center}
\caption{Format of TTS supplied detector data}
\label{tab:dataformat}
\end{table}

Thus each detector, named D\textit{N}, has a column where the numbers are the detected vehicles since last detection. The detections are made once for every cycle thus there will be some variations in the time between measurements due to the adjustments to cycle time, which are made by DOGS.

It was agreed with DRD that the temporal resolution be fixed at 15 minutes. This accumulation was performed by indexing time within the data period into 15 minute bins and summing detections from the last 15 minutes.

Multiple data files - for different sets of detectors - was received for each area. Since the data period was not identical over the data files it was necessary to perform some cleanup in order to generate accumulated data in which all detectors were represented.

\begin{table}[!ht]
\begin{center}
\begin{tabular}{c|c|c|c}
\textbf{Area} & \textbf{From} & \textbf{To} & \textbf{Detectors} \\ \hline
Herlev & 13-11-2007 20:45 & 28-11-2007 09:45 & D3-D8, D13-D15, D18 \\
Glostrup & 13-11-2007 08:45 & 26-11-2007 14:15 & D1-D14 \\
\end{tabular}
\end{center}
\caption{Periods of aggregated and cleaned data sets}
\label{tab:dataperiod}
\end{table}

The cleaned data was exported to a single table of more than 30.000 rows in the format:
\begin{table}[!ht]
\begin{center}
\begin{tabular}{c|c|c|c|c|c}
\textbf{Date} & \textbf{Time} & \textbf{Area} & \textbf{Detector Name} & \textbf{From Direction} & \textbf{Vehicles} \\
\end{tabular}
\end{center}
\caption{Format of cleaned data set}
\label{tab:cleandataformat}
\end{table}

In the next section I will use the cleaned data to make some analyses and comparisons of the areas to discover facts on \textit{directional proportions} and \textit{distribution of traffic on a daily basis}.

\subsection{Detector Data Analysis}

The first four graphs (Figures \ref{fig:herlev_props_morning}-\ref{fig:glostrup_props_afternoon}) shows the overall fluctuation of traffic (mean-value) on workdays and in weekends split on mornings (7-9) and afternoons (15-17).

\begin{figure}[ht]

    \begin{minipage}[b]{0.5\linewidth}

\begin{center}
\includegraphics[scale=0.25]{herlev_direction_proportions_morning.png} 
\end{center}
\caption{Herlev - morning}
\label{fig:herlev_props_morning}

    \end{minipage}
    \hspace{0.5cm}
    \begin{minipage}[b]{0.5\linewidth}

\begin{center}
\includegraphics[scale=0.25]{herlev_direction_proportions_afternoon.png} 
\end{center}
\caption{Herlev - afternoon}
\label{fig:herlev_props_afternoon}

    \end{minipage}

\end{figure}



\begin{figure}[ht]

    \begin{minipage}[b]{0.5\linewidth}

\begin{center}
\includegraphics[scale=0.25]{glostrup_direction_proportions_morning.png} 
\end{center}
\caption{Glostrup - morning}
\label{fig:glostrup_props_morning}

    \end{minipage}
    \hspace{0.5cm}
    \begin{minipage}[b]{0.5\linewidth}
    
\begin{center}
\includegraphics[scale=0.25]{glostrup_direction_proportions_afternoon.png} 
\end{center}
\caption{Glostrup - afternoon}
\label{fig:glostrup_props_afternoon}

    \end{minipage}

\end{figure}

We can see that there is an overweight of northgoing traffic. This overweight is more profound in the Glostrup area and increases in the afternoon for both areas.

From these graphs we can also see that, in the weekend, most traffic happens in the afternoon, since most danes go to church in the mornings.

The next graph, see Figure \ref{fig:herlev_commuter}, shows the usual commuter- and lunch traffic patterns. The data from all mondays to fridays in the dataset have almost identical temporal distributions and thus the graph shows summarized data from workdays.

\begin{figure}[!ht]
\begin{center}
\includegraphics[scale=0.4]{herlev_workday_distribution.png} 
\end{center}
\caption{Distribution of traffic throughout workdays in Herlev}
\label{fig:herlev_commuter}
\end{figure}

By comparing the workday distribution to the detected traffic in the weekend, see Figure \ref{fig:herlev_weekends} - and at the same time looking and the proportions of demand in Figure \ref{fig:herlev_props} - it is clear that O3 in Herlev is heavily used by commuters. Traffic is almost identically distributed on saturdays and sundays and the graph shows summarized data for the weekend.

\begin{figure}[!ht]
\begin{center}
\includegraphics[scale=0.4]{herlev_weekend_distribution.png} 
\end{center}
\caption{Distribution of traffic throughout the weekend in Herlev}
\label{fig:herlev_weekends}
\end{figure}

As the directional proportions remain the same as in Figure \ref{fig:herlev_props}, the last two graphs summarize on direction.

The next graph (Figure \ref{fig:herlev_southgoing}) show how detections are aligned in the southgoing direction. 

\begin{figure}[!ht]
\begin{center}
\includegraphics[scale=0.4]{herlev_southgoing_detectors.png} 
\end{center}
\caption{Relative detections for southgoing links in Herlev}
\label{fig:herlev_southgoing}
\end{figure}

With no intermediate links the lack of turn in- and turn out traffic would cause the detections to be evenly distributed. However we see, for example, that from D3 to D4 there is a major increase in traffic. This is caused by traffic turning onto O3 in the southgoing link from Herlev Amtssygehus and as such it would be valuable to have a dedicated detector for this link.

\begin{figure}[!ht]
\begin{center}
\includegraphics[scale=0.4]{herlev_northgoing_detectors.png} 
\end{center}
\caption{Relative detections for northgoing links in Herlev}
\label{fig:herlev_northgoing}
\end{figure}

In Figure \ref{fig:herlev_southgoing} - the northgoing direction - we see a bump in traffic between detectors D13 and D17 from 18\% to 22\%. Between these lie Herlev Bygade and, in particular, Herlev Hovedgade. From D5 to D18 there is a 2\% increase primarily due to traffic from the Hospital.

(For either direction these proportions do not vary in particular between workday and weekend.)

\subsection{Glostrup}
This area is better covered with respect to detectors and we even have data from detectors on the "minor roads" at roskildevej.