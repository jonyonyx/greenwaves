\section{Introduction}

The ringroads sorrounding the city of Copenhagen serve to route traffic from east and west around the city. As such they are part of the collection of arteries in Denmark.

The DOGS system by Technical Traffic Solution (TTS) was chosen to adjust the capacity of ringroad 3 due to the rebuilding of the nearby motorring 3, which was expected to cause increased deman on ringroad 3.

DOGS increases capacity by simultaneously increasing cycle times in all signal controllers, when certain traffic conditions arise. These criteria are determined statically by so as to alleviate the most heavily loaded intersection in the DOGS area.

The purpose of this study is to simulate DOGS to discover the true effects of the system. In previous analysis by the Danish Road Directory (DRD) certain analytical observations have been made, which indicate that DOGS disrupt coordination on ringroad 3.

The simulation tool is chosen to be Vissim, which is the de-facto microsimulation tool in denmark. Ringroad 3 consit of two DOGS areas, which are separated by three intersections, and due to the combined size of the network tools are developed to insert link inputs and route choices.

To test the issue of disrupted coordination during DOGS operation an offset optimization tool was developed to provide precalculated offsets for each signal controller for each cycle time. This tool also integrates closely with Vissim to extract informations such as distances and signal controller plans. 