\section{Introduction}

Road traffic has become an essential part of modern society and is putting an increasing
demand on road networks.
Traffic network congestion causes delays which add substantial costs
to society and businesses on a daily basis and also increase emissions
and the risk of accidents.

To alleviate congestion, public transport can be improved or the
infrastructure can be expanded. In urban areas, the latter is often
impossible due to residential areas adjacent to the existing roads.  A
more subtle way to improve the network performance is to make better
use of the existing roads, which can be achieved in part by proper
setting of traffic signal parameters.

It is estimated that the proper use of intelligent traffic systems
including intelligent traffic signals, could increase the capacity of the
road network in the Greater Copenhagen area by 5 to 10\% and in the report \cite{bedresignaler} simulations reveal that optimized coordinations for a circular artery in Copenhagen can reduce delays and stops in the morning rush hour by more than 25\%, compared to current settings.

The ringroads sorrounding the city of Copenhagen are part of the collection of arteries in Denmark and serve to route traffic around the city to offload the urban road networks and provide an alternative to going through the city. As such the signal controller settings are frequently adjusted in an offline process (see eg. an optimization using TRANSYT\footnote{See section \ref{simulation} for more details on simulation	tools.} \cite{transyt}) to best serve the demand from road users. Offline signal settings are made for some prespecified time interval eg. "morning" or "afternoon" and cannot compensate for the dynamic aspects of traffic, unlike adaptive signal control systems.

The DOGS system by Technical Traffic Solution (TTS) is a system for intelligent traffic light management on an artery. DOGS was chosen by the county of Copenhagen to adjust the capacity of ringroad 3 due to the rebuilding of the nearby motorring 3, which was expected to cause increased traffic.

DOGS increases capacity by simultaneously increasing cycle times in all signal controllers, when certain traffic conditions arise. These criteria are determined statically so as to alleviate the most heavily loaded intersection in the DOGS area.

The purpose of this study is to simulate DOGS to discover the true effects of the system. In previous analyses by the Danish Road Directorate (DRD) certain analytical observations have been made, which indicate that DOGS is capable of disrupting coordination on ringroad 3.

The simulation tool is chosen to be Vissim, which is the de-facto microsimulation tool in Denmark. Ringroad 3 consist of two DOGS areas, which are separated by three intersections, and due to the combined size of the network a code library is developed for Vissim to perform tasks such as inserting link inputs and route choices from traffic counts and running tests with data extraction.

To test the issue of disrupted coordination during DOGS operation an offset optimization tool was developed to provide precalculated offsets for each signal controller for each cycle time. This tool also integrates closely with Vissim to extract information such as distances and signal controller plans. 

The report is structured as follows. After introducing traffic signal optimization systems in section \ref{systems} I describe how DOGS work and what the intentions behind DOGS are in section \ref{dogs}. Section \ref{simulation} introduces the Vissim microsimulator and I describe how Vissim structures its data and how I can take advantage of the plain text property of the Vissim network file to automatically insert traffic data for link inputs and route choice. In section \ref{vap} I describe the Vehicle Actuated Programming language (VAP) of Vissim and how I use it to emulate DOGS in a master-slave scheme. For the purpose of adjusting the simulation I received detailed traffic data from both the DRD (traffic counts and signal layouts) and from TTS detector data. These data are analysed in section \ref{data} where I show the arterial nature of ringroad 3 and other properties such as direction bias. The Vissim network I use in this project was started by COWI\footnote{A Danish consulting firm in engineering, environmental science and economics.} and later inherited and improved by many students at the Technical University of Denmark. In the next section \ref{modelling} I discuss how I expanded and modified the network using automatic procedures, which work directly on the data structures described in section \ref{simulation}. Section \ref{optimization} is where I discuss optimization of coordination and how I designed my own system based on simulated annealing. The last section \ref{test} compares the performance of original DOGS and DOGS with offsets from the optimizer to the basic program. Finally in section \ref{conclusion} I bring my conclusions and suggestions for future works based on the results of section \ref{test}.

\subsection{Terms}
\label{vocabulary}
When I first started studying the literature on traffic signal optimization \cite{forprojekt} it became evident that there was a great deal of terminology specific to the field of traffic signal settings and that most articles assumed the author to be familiar with it. 

It is my impression that the terms of traffic signal optimization are fairly standardized and most articles will share terminology.

This section attempts to extract the most important terms and give solid descriptions, so that the field can be adopted by newcomers more quickly.

\begin{description}

\item[Artery] A main-path, the major road, through a traffic network. It will generally face higher demand than minor roads adjoining the artery.
			
\item[Coordination] Especially relevant to arteries, the quality of coordination between signal controllers determines how road users perceive a journey through an artery. With good coordination, the platoons of vehicles will experience a green light whenever they approach the next intersection, this is also known as the \textit{green wave}.
			
\item[Cycle time] The turnaround time for all phases of a traffic signal to complete ie. the time it takes from the start of green time for a phase until it becomes green again. A \textit{common} cycle time is especially relevant for the signal controllers in an artery in combination with proper offsets to establish good coordination.

\item[Green Wave] Road users experience green waves when they receive green every time they reach the next intersection. A progression band is a graph in time and distance describing the progression of a platoon of vehicles. In combination with the states of the signal controllers progression bands form road-time diagrams that give an impression of the quality of such green waves.

\item[Interphase green] Also known as lost time, is a small amount of time inserted as a buffer between two phases. During the lost time the lights can be either red in all directions or, as in Denmark, amber lights can be used to introduce a buffer. The purpose of the buffer is to allow vehicles, which entered during the last phase, to exit the intersection before it is flooded by vehicles from the next phase.

\item[MOE] An abbreviation for Measure Of Effectiveness and also referred to as the performance index (PI) or fitness. MOE is some metric on which the performance of a traffic signal network is assessed .
	Most often used is the average delay, also common is the travel time through the network and number of stops or some combination.
	
\item[Offset] Only relevant under cycle-time-based programs, the offset is the delay with which to start the execution of the signal program, relative to the master controller. Offsets are chosen for each signal controller in a way such that good coordination is acheived. The cycle time must be common to all signal controllers otherwise the offset only provide the expected coordination in the first cycle and periodically in the later cycles.
	
	\item[Phase] Sometimes referred to as \textit{stage}, corresponds to a particular combination of the red and green lights of the signal heads in an intersection.
	For instance there may be a phase of green in the north and south direction for a two-way intersection (which implies red lights are shown in the east-west direction).
	
	\item[Platoon] A group of vehicles travelling together. A platoon can be detected by observing the time between a vehicle and the next and applying a threshold in time units known as the critical headway.
Platoons are formed both as a consequence of car-following behaviour, which is used in simulation frameworks such as \cite{treiber-2000-62} and Vissim, but also due to the batch-like nature which is imposed on the traffic by traffic signals.
Platoons are \textit{dispersed} ie. split up over time into multiple platoons due to the individual behaviourial elements (eg. desired speeds) of road users.

\item[Queue spillback] This phenomenon occurs when a queue reaches from a downstream intersection to the preceding intersection, effectively preventing traffic from leaving the upstream intersection.

\item[Signal Controller] A means for controlling the right-of-way of conflicting traffic motions in an intersection between two or more roads. Right-of-way is periodically shifted between incompatible traffic flows by choosing one signal group after another.

\item[Signal Group] A collection of signal heads, which show identical colors at all times.

\item[Signal Head] A traffic light, which constitute a signal group itself or, more commonly, is a part of a signal group.

\item[Signal Program] A description of the states of the signal groups in the course of a cycle. The signal program is repeated after each cycle completion.

\item[Time horizon] The amount of time, which is taken into consideration while optimizing signal settings or making predictions. Since predictions of the future traffic becomes more and more fuzzy the deeper one looks a paradox arises: using a short time horizon the optimizations might prove to be flawed when it fails to see a clever decision, but with longer time horizons the predictions themselves become flawed and may mislead the optimization.

\item[Traffic assignment] Also known as flow assignment, is the determination of vehicular flow along origin-destination (OD) paths and, consequently, along links in a traffic network. Traffic assignment is in contrast to static assignment, when link inputs and routes are fixed before the simulation starts.

\item[Traffic network] A graph $G(V,E)$ where $V$ is a set of intersections controlled by a traffic signal and $E$ is the set of roads connecting the intersections. A path is thus a route through the network crossing a least one signalized intersection.

\end{description}

\subsection{Software versions}
The Vissim network was developed first under Vissim 4.00-16 and later in the newly released Vissim 5.00 using the latest service packs as they were released. All tests were performed in Vissim version 5.00-08, however. 

Libraries for Vissim manipulation, support and optimization were written in Ruby, targeting version 1.8.6.
