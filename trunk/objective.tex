\label{scope}
The objective of this survey is to cover the research into adaptive traffic control with emphasis on the applied optimization methods.

The survey was made from a collection of about 50 papers and research reports, which cover from static optimization of a single intersection to large-scale systems designed to coordinate multiple signals. 

In addition, to obtain some operational insight, two interviews was performed with danish road authorities. 

The first one was with \textit{Lars Bo Frederiksen} and \textit{Nicolai Ryding Hoegh} of Centre for Traffic. They control more than 350 signals in the municipality of Copenhagen including 2 systems for adaptive control (in Valby and at \textit{Centrumforbindelsen} resp.). 

The second interview was done with \textit{Steen Merlach Lauritzen} at the Danish Road Directorate. The Danish Road Directorate supervise the regional road infrastructure which is mostly freeways and highways. They have a number of adaptive control systems (MOTION by Siemens, UTOPIA/SPOT by Swarco, DOGS by Technical Traffic Solution).

This report will discuss such things as:

\begin{itemize}
\item The objective function and the inherent multi objectivity of the problem
\item Optimization types
\item Cooperation among signal controllers
\end{itemize}

It will not go into detail on e-conomic or safety issues nor will it discuss optimizations for soft trafficants. Detection is not investigated in detail; it is assumed that there exist solid solutions for detection and that the control software has good to near-perfect information about the state of the traffic network.
