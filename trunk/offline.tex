\label{offline}
The surveyed articles span the period from the 60's to present. In this section some tendencies for offline optimization systems are surveyed.

Most authors choose to present a mathematical model of the problem, in the spirit of the one presented in section \ref{model}, which optimizes for one or more of the parameters mentioned in section \ref{theproblem}. Attempts to produce a closed form solution for the problem is seen often eg. \cite{36}, which presents a general model that has resemblences with the model in section \ref{dynamicmodel}.

For representation of network layout the models are based on either graph theory or some kind of cell transmission scheme or petri net system. The petri net representation is very popular and some examples of its use are \cite{12}, \cite{16} and \cite{petri}.

In a petri representation of a traffic network spaces on links are represented by cells, which can fit a single vehicle. When a cell is occupied no other vehicle may enter. Vehicles progress through the network by making transitions to adjacent cells. The stop-bars of intersections can be represented in the petri net as a line of cells blocking a link facing the intersection. When the phase begins the stop-bar cells are enabled and vehicles may pass.
The petri net representation fits well into the context of traffic networks due to its representation of concurrency (spaces in a lane) and triggered actions (cross on green light).

For the model to be solvable without an external evaluation tool, such as TRANSYT, it must contain functional constraints which relates eg. the proper offset for signals along an arterial to flow specific parameters such as saturation flow rate and the platoon dispersion factor.
The model must also adjust link flows according to the signal settings that is, the model must be bilevel, as introduced in section \ref{bilevel}. An example of such a model is seen in \cite{33} where a genetic algorithm finds optimal signal timing plans, considering user response to signal changes. TRANSYT is used to obtain a fitness (PI) value and the Path Flow Estimator\footnote{Developed by Transport Operation Research Group in Newcastle University} for determination of the stochastic user equilibrium.

The inherent bilevel nature of the problem requires either a single, complex, optimization routine which considers the signal setting and equilibrium problem at once or an overall routine which performs iterations of signal setting optimization and equilibrium determination until some threshold for agreement is reached. 

Considering a typical optimization formulation involving a common cycle time, green splits and offsets it is clear that - even with some dicretization of time - the search space is vast, especially when considering some form of simulation for obtaining the objective value. An example of a heuristic designed to cut down on the search space is ADESS (see \cite{26}). Most often the heuristics are embedded into the search routines and operate, for instance, by using "common sense" rules such as pruning of short cycle lengths under high saturation or by exploting sensitivity knowledge of the problem \cite{40}.
Another way to deal with this issue is to employ a metaheuristic search, which is simply run until a result is needed. Some examples are: \cite{1} (tabu search), \cite{42} (particle swarm optimization) and genetic algorithms, which are the most popular metaheuristics, by far see \cite{13}, \cite{33}, \cite{43}, \cite{7}, \cite{41}, \cite{31}, \cite{27}, \cite{2} and \cite{26}.

Systems which operate in this manner are mostly suited for offline use considering that some reported running times are as high as several hours even for small sized networks.
Some authors eg. \cite{16} turn such a system into an adaptive system by re-running the optimization procedure every $K$ cycle but clearly this strategy is not possible for systems with long run time requirements. Furthermore the next set of signal timing settings may be quite different from the current one and changing plans in an instant will cause transient sideeffects such as malfunctioning green waves.

Much work focus on optimizations for single intersections or coordination of signals along an arterial. Gartner et al. \cite{9} gives a walkthrough of the most promising progression schemes at the time (PASSER-II, NO-STOP-1) and extend the MAXBAND approach by Little, Kelson and Gartner (1981) from a single-arterial bandwidth optimization into MULTIBAND, which optimizes bandwidth along multiple, possibly intersecting, arterials. \cite{6} is an example of the selection and optimization of an arterial - or priority route - with subsequent network wide optimization, taking the arterial optimization as a constraint.
In the article \cite{24} Heydecker propose how to use existing optimization methods for single intersections in combination to achieve a network optimization. Though Heydecker admits that, because of the decentralization, it will be difficult to obtain coordination.