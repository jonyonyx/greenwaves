The surveyed articles span the period from the 60's to present. There are some tendencies which has been observed with respect to optimization methods.

A pattern for the earlier work seems to be the presentation of a mathematical model, similar to the one presented in section \ref{model}, which optimizes for one or more of the parameters mentioned in section \ref{theproblem}. 

The model will contain functional constraints which represent the relation of eg. the proper offset for signals along an arterial given such parameters as saturation flow rate and platoon dispersion factor.

The problem is then shown to be time too consuming to solve using current computers and a heuristic is presented, which can cut down on the search space eg. by using "common sense" rules such as pruning of short cycle lengths under high saturation.

Later on the same models begin to become solvable by standard optimization packages due to advances in computer power as well as in the packages themselves. Most work until then has focused on optimizations for single intersections or coordination of signals along an arterial. Gartner et al. \cite{9} gives a walkthrough of the most promising progression schemes at the time (PASSER-II, NO-STOP-1) and extend the MAXBAND approach by Little, Kelson and Gartner (1981) from a single-arterial bandwidth optimization into MULTIBAND, which optimizes bandwidth along multiple, possibly intersecting, arterials. 

