\section{Signal Control Systems}
The systems that mandate road use by signal controllers can roughly be divided into two major types: adaptive and non-adaptive. Within each type the systems are specialized with respect to the number of signal controllers the system governs and the network layout, when more than one intersection is under signal control. These specializations are either for arterials or the general network.

This section introduces briefly the major trends in these two systems to outline the setting in which DOGS comes in.

\subsection{Traditional signal control}

The traditional systems, which are also the simplest, operate on the basis of signal plans, such as the ones seen in appendix \ref{app:signalplans}. For isolated intersections, there is a choice between pretimed and traffic actuated control strategies. Pretimed control involved the use of static signal plans for phase sequences, cycle time and green splits according to the day of time. Traffic actuated signals are usually flexible versions of the static signal plans, allowing the green time of stages to be extended, up to some maximum when vehicles are detected.

When more than one signal controller is under system control the signal plans are based on a common cycle time and the signal controllers are adjusted relative to each other by choosing an offset, for each signal controller, which provides good coordination. Traffic actuated signals cannot cooperate with other signal controllers in an artery since they operate on a variable cycle time depending on the amount of time each stage was extended.

Pretimed control is are based on the assumption that demand is fairly stable within certain divisions of time eg. morning, midday and evening or workday \/ weekend. For instance, in the morning (7.00 to 9.00) and afternoon (15.00 to 17.00) the traffic is usually heavier than during the day or night due to the commuters. The exception is in the weekend, where increased demand scenarios occur before eg. a football game starts.

\subsection{Adaptive signal control}

The dynamic aspects of traffic are obvious when considering the things that affect it:

\begin{itemize}
\item Events: football games, lane closures due to VIP transport etc. will cause a focusing of traffic in certain areas
\item Weather: warm weather will cause more traffic toward beach areas, winter and snow causes reduced speeds but potentially also fewer vehicles on the road
\end{itemize}

It is near impossible to take into account such phenomenon when designing pretimed signal plans. In fact traffic engineers will attempt to capture the least "eventful" (neutral) dataset possible when designing such plans to accomodate the most common traffic situation.

Adaptive signal control adjust signal controllers in an arterial or network to coordinate intersections so as to optimize some performance index eg. average delay or number of stops (or a combination) but also to reduce the need for constant supervision and tuning of intersections.

This possible by dynamically adjusting cycle times, phase sequences and green times according to detected as well as predicted traffic and thereby reacting to those dynamic aspects of traffic, which cannot be captured by the static optimization routines used to generate time-of-day plans. Some systems (\cite{phase_by_phase}, \cite{rhodes}, \cite{opac}, \cite{scoot}) even skip or work around the conventional periodic scheme based on a common cycle time and make direct assignments of phases and allow phases to be skipped, as discussed later in section \ref{phase_based}. 

The key to adaptive signals is reliable detection and short term prediction of traffic. Adaptive signals must be able to respond to the dynamic aspects of traffic, which are not captured in the design of pretimed signal plans. Adaptive signals use historical input and current detector input to make short term predictions for what is going to happen eg. within the next minute, next 10 minutes and so forth.

An isolated adaptive signalized intersection has advantages over actuated signals because it can skip a phase to give priority to a bus, for instance. The main attraction with adaptive signals is when they can be set to work together. A good system will naturally cause green waves to appear and move the direction of the green waves along with changes in flow.

It is evident that the cycle time is crucial in optimization because, for a congested network, increasing the cycle time will cause increased capacity and - hopefully - increased throughput as well. Long cycle times lead to long phase durations, which allow a steady flow of vehicles to pass and minimizes lost and interphase time per time unit.

DOGS is a hybrid somewhere between adaptive and traditional pretimed systems for arterial signal control. DOGS has no prediction capabilities, but relies alone on detector values, and is thus lacking a key component of a other adaptive systems. 

This, however, gives DOGS a clear advantage in optimization since, for each cycle, the only decision to be made is whether to remain at the current level (ie. common cycle time) or to change the level by one ie. increasing or decreasing the common cycle time. This decision can be made using little computational effort (see section \ref{dogs}) and this, along with the immediate adoption of the preexisting signal timing plans, makes DOGS a simple and easily implemented arterial optimization system.

DOGS has been implemented on several arterials in Denmark. It relies on the site specific knowledge of traffic engineers for definition of criteria parameters, which must be gathered in a manual process and reevaluated over time. DOGS is different from other adaptive systems by the fact that it only operates under medium to heavy traffic loads under which the problem of choosing signal settings reduce to the proper setting of capacity on the major road and coordination by offsets.