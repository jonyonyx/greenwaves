\section{Test Scenarios}
The purpose of this section is to highlight the findings in the various simulation runs which were performed.

The primary purpose of the simulation was to test the DOGS system against a theoretically better founded system for dynamic signal control. In addition tests were performed to discover the effect of bus priority and to determine whether DOGS with its varying cycle time can outperform a fixed cycle time but properly coordinated offsets.

All simulations were fitted to the traffic observed in the morning period fra 7-9 am.
To account for fluctuations each test was run 10 times with different seeds.

The results of the tests are measured by extraction of average queues and delays for each intersection. This is done by inserting a node with node evaluation enabled for each of the twelve intersections.

\subsection{The Effect of DOGS}
This test covers four scenarios:

\begin{enumerate}
\item DOGS and bus priority
\item DOGS without bus priority
\item Default program with bus priority
\item Default program with no priority for buses
\end{enumerate}

In scenario 1 and 2 DOGS is enabled and the capacity in the main direction is adjusted according to detector values. In scenarios 3 and 4 DOGS is disabled effectively ignoring all detections (except for buses in scenario 3) and thus the signals will remain in the base program.

As DOGS will naturally cause decreased performance for the minor roads it was decided to split the dataset so that \textit{arterial} and \textit{crossing} traffic can be distinguished. For delays arterial traffic is defined as traffic which enters an intersection from north or south and makes a throughgoing motion rather than turning off the arterial. All other traffic is marked as crossing.

\subsection*{Bus Priority}


\subsection{Coordination with Direction Priority Offsets}
In this test the system for coordination of section \ref{coordination} os compared with DOGS without bus priority. DRD requested this test to discover whether DOGS (dynamic capacity) can outperform a proper coordination. DOGS has support for dynamically adjusting the coordination - including prioritizing directions - though it is not enabled in either the Glostrup or Herlev sites at the moment.

