\section{Conclusion}
\label{conclusion}

The DOGS system for arterial traffic optimization was implemented in the Herlev and Glostrup sections of ringroad 3 in response to the expansion of the nearby motorring 3 due to expected traffic increases as an effect of the work.

DOGS was developed by danish Technical Traffic Solution and basically uses threshold values to determine a proper common cycle time so that increase in load on arterial automatically causes the capacity to increase when the extra green is primarily given to the stages accomodating arterial traffic.

However DOGS was never tested in a simulation environment, such as the (in Denmark) de-facto Vissim simulation tool. Vissim is a microsimulator well-suited for testing complex situations and signal controller configurations.
The Danish Road Directorate (DRD) suggested that DOGS be tested in Vissim to discover if the system delivers the necessary offload, when traffic is diverted onto ringroad 3.

In this report are the results of such a test performed for the morning period in the two disjoint DOGS areas, Herlev and Glostrup, consisting of 5 and 4 intersections, respectively. 

The results show that DOGS, although capable of smoothening the peak traffic periods, does not convicingly improve on arterial traffic conditions when compared to the preexisting, pretimed signal plans. The reason, as previously speculated by DRD, is traced back to the uncontrolled of change green time displacement, when the cycle time is increased without choosing new offsets. This result is found by comparing DOGS with a modified version, which does implement optimized offsets for each common cycle time.

The optimization routine used to generate these offsets is based on the metaheuristic optimization scheme simulated annealing, a hill climber with the ability to escape local optima. The evaluation criterion is directly derived from the same analysis, which raised questions concerning the coordination properties of DOGS ie. the green time displacement.

Accepting the de-facto status of Vissim the system extracts information from the actual Vissim network, wherever possible, reducing the need to redesign a network in another tool eg. TRANSYT. Such information include signal controller stages and signal timing plans and distances between intersections.

Features, which have been built into the optimization system, include prioritizing coordination for a specific direction bias, achieving better coordination through speed adjustments for individual stretches between intersections and emphasis on good coordination between close-by intersections. The system makes use of random restarts, focus and reheatings to effectively traverse as much of the search space as possible.

In addition to testing DOGS, the bus priority, which is present for the three northern-most intersections of Glostrup, was tested and found to give small but consistent reduction in average delay for the buses using the arterial.

\subsection{Future works}
I strongly believe per-level offsets should be further considered in the next revision of DOGS. In conjunction average speeds between intersections and hence travel times used in the optimization should be considered. It is safe to assume that the traffic conditions, which trigger each DOGS level, can be mapped upon a certain set of travel times. I expect the travel time to increase in some proportion to the proper DOGS level and this information can be used to further improve coordination, as the results of this study assumed fixed travel times.

The optimization routine was implemented in the interpreted language Ruby. Although the implemention will try more than 1000 combinations per second on a 2 GHz CPU, it is expected that an implementation in a faster language, such as C\# will lead to even better solutions.

As mentioned the implemented system is capable of optimizing coordinations by changing speeds as well as offsets. Dynamic speed signs are becoming more and more widespread with the introduction of LED technology, for instance on motorring 3, and simulation studies should be performed to test wether road users will be able to respond to such speed signs indicating \textit{recommended speed for green wave}.
