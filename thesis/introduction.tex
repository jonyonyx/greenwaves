\section{Introduction}

The ringroads sorrounding the city of Copenhagen serve to route traffic from east and west around the city. As such they are part of the collection of arteries in Denmark.

The DOGS system by Technical Traffic Solution (TTS) was chosen to adjust the capacity of ringroad 3 due to the rebuilding of the nearby motorring 3, which was expected to cause increased deman on ringroad 3.

DOGS increases capacity by simultaneously increasing cycle times in all signal controllers, when certain traffic conditions arise. These criteria are determined statically by so as to alleviate the most heavily loaded intersection in the DOGS area.

The purpose of this study is to simulate DOGS to discover the true effects of the system. In previous analysis by the Danish Road Directory (DRD) certain analytical observations have been made, which indicate that DOGS is capable of disrupting coordination on ringroad 3.

The simulation tool is chosen to be Vissim, which is the de-facto microsimulation tool in denmark. Ringroad 3 consist of two DOGS areas, which are separated by three intersections, and due to the combined size of the network tools are developed to insert link inputs and route choices.

To test the issue of disrupted coordination during DOGS operation an offset optimization tool was developed to provide precalculated offsets for each signal controller for each cycle time. This tool also integrates closely with Vissim to extract informations such as distances and signal controller plans. 

The report is structured as follows. In section \ref{dogs} I formally describe how DOGS work and what the intentions behind DOGS are. Section \ref{simulation} introduces the Vissim microsimulator and I describe how Vissim structures its data and how I can take advantage of the plain text property of the Vissim network file to automatically insert traffic data for link inputs and route choice. In section \ref{vap} I describe the Vehicle Actuated Programming language (VAP) of Vissim and how I use it to emulate DOGS in a master-slave scheme. For the purpose of adjusting the simulation I received detailed traffic data from both the Danish Road Directorate (traffic counts and signal layouts) and from TTS detector data. These data are analysed in section \ref{data} where I show the arterial properties of ringroad 3 and the direction bias. The Vissim network I use in this project was started by COWI and later inherited and improved by many students at Technical University of Denmark. In the next section \ref{modelling} I discuss how I expanded and modified the network using automatic procedures, which work directly on the data structures described in section \ref{simulation}. Section \ref{optimization} is where I discuss optimization of coordination and how I designed my own system based on simulated annealing. The last section \ref{test} compares the performance of original DOGS and DOGS with offsets from the optimizer to the basic program. Finally in section \ref{conclusion} I bring my conclusions and suggestions for future works based on the results of section \ref{test}.

All tests were performed in Vissim version 5.00-08. The support and optimization programs were written in Ruby version 1.8.6.