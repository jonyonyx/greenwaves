\section{Simulation \& Vissim}
\label{simulation}
Traffic simulation is the emulation of traffic on a computer and it is done to gain insight in road user behaviour and observe various effects eg. what happens to the minor-road traffic when we increase the cycle time to 100 seconds.

A main issue of traffic signal optimization is that of how to evaluate solutions. Fitness determination of a set of signal parameters is a major issue, especially in metaheuristic search, which is prevalent in the litterature and in this project, that put "blind faith" in the evaluation procedure. This is to be understood such that metaheuristic search routines do not necessarily search the solution space in a manner in which each step a better solution is obtained, but rather by trial-and-error. 
Furthermore the metaheuristic must evaluate a certain amount of solutions before confidence in the final solution is established, thus each evaluation must be fast if the search is a part of an adaptive system, rather than an offline optimization that has no strict deadline.

Simulation handles the complexity issues of modelling the bilevel system of mutually responsive road user behaviour versus traffic signal settings and allows the optimization routine to rely on the simulation results as a black box for fitness evaluation. 

The primary use of simulators are for infrastructure expansions and testing various signal timing plans, of course. In traffic signal optimization they are almost always used to establish confidence in performance for some signal optimization procedure. This confidence comes from observing the actions of the system during the simulation but also by comparing aggregated fitness values with those coming from some other, well-known approach. In this regard the TRANSYT offline traffic signal optimization package is often used as a baseline. 

This form of testing favors quality of the simulation over execution speed, unlike search routines, which must have results quickly. 
In Table \ref{tab:convergespeed} are the names of three major simulators, which are used by traffic engineers to produce realistic traffic simulations. 
In Denmark the de-facto simulation tool for final testing of signal settings and infrastructure changes has become Vissim by PTV. For obtaining signal settings TRANSYT is highly regarded by DRD though the results will often be adjusted and confirmed in Vissim.

There are three types of simulators, the main difference being the level of detail in the underlying model. They are micro-, meso- and macrosimulators. Microsimulators model the behaviour of each vehicle and driver, mesosimulators regard the movements of platoons of vehicles and macrosimulators search for equilibriums between signal settings and road user response (route choice), considering origins (input links) and destinations (output links).

Independent on the detail level of the model a simulation must run until it is converged before the fitness can be determined. 

Previous studies (see \cite{corsimvstransyt}) have shown that the CORSIM microsimulator is superior to the mesoscopic simulator from the TRANSYT package when combined with a genetic algorithm, since the flow equations do not capture the dynamic aspects of traffic. But microsimulation is also slower to converge and thus evaluation of a solution will take longer. In a comparison study \cite{simcompare} Holst estimates the required relative simulation time in a large network, see Table \ref{tab:convergespeed}.

\begin{table}[ht]
\centering
\begin{tabular}{c|c|c}
\textbf{Vissim} & \textbf{Dynameq} & \textbf{Time Slice Assistant} \\
\textit{(micro)} & \textit{(meso)} & \textit{(macro)} \\ \hline
1000 & 100 & 1
\end{tabular}
\caption{Estimate of of relative required simulation time for convergence. Convergence of a Vissim simulation takes approximately 1000 times longer then the same simulation in TSA.}
\label{tab:convergespeed}
\end{table}

Microsimulation will not scale as well as meso- and macrosimulation since the number of vehicles in the network will grow in proportion with the total length of the roads. In mesosimulation the number of platoons need not grow as fast since some arteries traverse the entire network and a single platoon is regarded. The headway threshold for platoon definition could also be increased to compensate for the extra cars. Macrosimulators will scale in proportion with the size of the OD-matrices, which is the minimal level of detail and complexity for realistic results. 
Meso- and macro simulators, using deterministic principles, have another advantage since they can execute heuristic procedures to skip past much of the initial population of roads, which is mandatory in microsimulation when traffic begins to flow into the empty network. Since most simulations will be stopped whenever convergence is reached this is a very useful attribute.

Aside from the commercial simulator packages, such as the ones listed in Table \ref{tab:convergespeed}, there are several examples of researchers implementing their own simulation tools. Such tools are tightly coupled to the test network and the results are questionable thus it is important that the industry agrees upon a common simulation platform, but also that common standards are used \textit{within} the agreed upon tool. Such a project has recently been undertaken by DRD to improve the comparability of results from different consulting firms.

\subsection*{Vissim}
\label{vissim}
Vissim provides a microscopic simulation of traffic in a network. In this context microscopic entails that each vehicle is individually modelled and, in a sense, Vissim takes the view of the users of the network.

The main feat of Vissim is the visual representation of vehicle movements and inspection of signal controller states. Furthermore Vissim allows a multitude of information to be extracted in addition to the ones listed below. This, and a theoretically well foundedness, is presumably why Vissim has become so widespread in Denmark.

\begin{enumerate}
\item Delay: the difference between the best-case travel time ie. no intersections and travel at free flow speeds (no other road users). The delay from point A to B when there are no intermediate intersections is an indication of the congestion
\item Travel time is measured as the raw time spent by road users getting from A to B in a network. Travel time sections can be placed anywhere and, like delay, can be filtered on vehicle type eg. car, truck and bus
\item Average queue lengths are correlated with delays especially when measuring delays for vehicles crossing an intersection
\end{enumerate}

Below I will briefly introduce the Vissim elements that have required the most attention in this project. In section \ref{modelling} I will go into the practical details of link inputs and route choices.

\subsubsection*{Network Elements}

Vissim is highly flexible with regards to network layout description and intersection turning movements can be fine-tuned almost to perfection, given enough time. The network is designed using a GUI interface where links are joined using connectors.

Links are directed sections of road with one or more lanes. All lanes have the same direction as the link - the opposite direction is modelled using a parallel link.

Connectors are mostly relevant in intersections where they connect exactly two links, however the lanes connected within the links can be chosen freely.

\subsubsection*{Traffic input}
Vissim is capable of dynamic traffic assignment (DTA) as well as static input. DTA is a method by \cite{Wardrop} in which the traffic flow adapts to the capacity of the network due to a \textit{stochastic user equilibrium} (SUE) ie. the road user response of traffic conditions model their choice of route. 

Vissim can perform this form of traffic input given zones, which are confined and non-overlapping areas of the network, and a matrix describing the relative traffic from each zone to each other zone.

In this project DTA was abandoned on recommendation from DTU Transport professor Otto Anker Nielsen due to the dynamic nature of the traffic signals since, in his experience, odd phenomenon may occur such as random gridlocks, when adjacent links are fully saturated.

Instead static input was chosen. In static input traffic enters on specific links and traverse the network until an exit link has been reached. Static input is used in combination with routing decisions, which are usually placed on input such that each vehicle entering will routed through the network according to the available traffic data.

\subsubsection*{Routing decisions}
\label{routingdecisions}
Routing decisions are made by vehicles when they pass over a routing decision point in the Vissim network. Selecting routes for vehicles can be done either by placing a routing decision point on an input link and designate a number of exit link options (end-to-end routes) or by placing routing decisions on each approach and route vehicles through eg. an intersection (local routes).

For the first option there exist many routes even for small networks when routes are symmetric (albeit using different links). Furthermore the number of routes will grow exponentially as the number of exit links (intersections) increase and the problem of mapping traffic count data onto end-to-end routes is not easy to solve. On the upside end-to-end routes are more correct to use since road users will most likely know how they wish to traverse the entire artery in advance and routes ending in an exit link is certain to cause no vehicles to cycle about indefinately in the network.

The second option requires a route per traffic stream in each intersection. Here a traffic stream is a pair of orientational markers indicating from where the route origins and where it ends eg. north-to-south (a through-going traffic stream) or north-to-west (a left-turning stream). This option is easy to implement given traffic count data since they immediately give the proportions of vehicles making each turn (ie. taking each route) in an approach. A vehicle will pass a number of routing decisions, obtaining new routes (making turning movements) until an exit link is reached.

It is possible to implement the first option by solving a linear program using the ordinary traffic counts as input data. The complexity of this solution caused me to adopt the second option instead and in section \ref{routefractions} I describe how I did this. Note that the problem of indefinite cycling the network cannot occur since I am regarding an artery only in this project and thus it is not possible to choose a cyclic route (there are no u-turns).
