\section{Introduction}
Road traffic has become an essential part of the modern society has put a high demand on the road networks supplied by the municipalities. 

Traffic network congestion causes delays which costs the society and businesses substantial amounts on a daily basis and also increase emmisions and the risk of accidents.

To alleviate congestion public transport can be improved or the infrastructure can be expanded. In urban areas, the latter is often infeasible due to residential areas adjacent to the existing roads. 
A more intricate way to improve the network performance is to make better use of the existing roads. This is where traffic signals comes into the picture. 

Traffic signals are employed in urban areas in order to control traffic flows and avoid collisions by fairly distributing the right-of-way for adjoining road segments and to ensure a smooth flow of traffic.

But if not set appropriately they may also seriously hinder the traffic flow through the road network and so great care should be taken when choosing the parameters for an intersection. 

The litterature has many suggestions for the intelligent setting of traffic signals, ranging from purely statistically based methods in the early 60's over actuated traffic signals to highly adaptive and cooperative methods, which can be realized using actual flow information coming from sensors. 

This report gives a survey of the litterature with special emphasis on adaptive methods, which attempts to coordinate traffic signals in larger networks so as to optimize some network-wide performance index. 

The report is structured as follows. In section \ref{vocabulary} the basic vocabulary of traffic control is summarized. Then the objective and scope of the report is stated in section \ref{scope}. 
The optimization problem and the traffic lights being optimized are described in sections \ref{theproblem} and \ref{signal_types}.
Section \ref{history} gives an outline of the research done in the traffic signal optimization. Then in section \ref{adaptive_cooperation} the attention is turned to adaptive systems with high degrees of cooperation.
Finally, in section \ref{conclusion}, are recommendations on methods and approaches for future work.