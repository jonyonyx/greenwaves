\label{scope}
The objective of this survey is to cover the research into adaptive traffic control with emphasis on the applied optimization methods.

The survey was made from a collection of about 50 papers and research reports, which cover from static optimization of a single intersection to large-scale systems designed to coordinate signals in multi-arterial networks. 

In addition, to obtain some operational insight, two interviews were performed with danish road authorities. 

The first one was with \textit{Lars Bo Frederiksen} and \textit{Nicolai Ryding Hoegh} of Centre for Traffic Copenhagen (CTC). They control more than 350 signals in the municipality of Copenhagen including 2 systems for adaptive control in Valby (MOTION) and at \textit{Centrumforbindelsen} (UTOPIA/SPOT) resp. 

The second interview was performed with \textit{Steen Merlach Lauritzen} at the Danish Road Directorate (DRD). DRD supervise the regional road infrastructure which is mostly freeways and highways. They have a number of adaptive control systems (MOTION by Siemens, UTOPIA/SPOT by Swarco, DOGS by Technical Traffic Solution).

MOTION in Valby receives data from 125 traffic detectors and use them to predict traffic within the next 15 minutes. For coordination it sets a common cycle time so that the intersection with the highest load can handle the demand.

In the hierarchical UTOPIA/SPOT system UTOPIA produces an overall plan for a network of SPOT controlled intersections. SPOT then makes fine adjustments respecting the constraints from the plan given by UTOPIA.

A detailed description of DOGS is found in section \ref{dogs}.

This report will focus on the following subjects:

\begin{itemize}
\item The objective function and the inherent multi objectivity of the problem.
\item Periodic and dynamic optimization strategies.
\item Cooperation among signal controllers.
\end{itemize}

It will not go into detail on economic or safety issues nor will it discuss optimizations for soft trafficants ie. pedestrians and bicyclists. Detection of traffic is not investigated in detail; it is assumed that there exist solid solutions for detection and that the control software has good to near-perfect information about the state of the traffic network.
