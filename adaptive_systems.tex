\label{adaptive_cooperation}
Adaptive traffic control systems aims to coordinate signal controlled intersections so as to optimize some performance index eg. average delay or number of stops (or a combination) but also to reduce the need for constant supervision and tuning of intersections.

They do this by dynamically adjusting cycle times, phase sequences and green splits according to detected as well as predicted traffic and thereby reacting to those dynamic aspects of traffic, which cannot be captured by the static optimization routines used to generate time-of-day plans. Some authors (\cite{1}, \cite{44}, \cite{46}, \cite{scoot2004}) even skip or work around the conventional periodic scheme based on a common cycle time and make direct assignments of phases and allow phases to be skipped, in the spirit of the dynamic model presented in section \ref{dynamicmodel}. 

It is evident that the cycle time is crucial in optimization because, for a congested network, increasing the cycle time will always cause a througput increase (there are always cars waiting to cross the intersection). In the litterature the cycle time is often common to all intersections under traffic control so that green waves (arterial progression) can be produced. For direct assignment systems the throughput is dependent on allocation to phases of consecute green time. Long cycle times lead to long phase durations, which allow a steady flow of vehicles to pass and minimizes lost and interphase time per time unit.

For large networks the enforcement of a common cycle is inappropriate, however. Consider a network which is so large that two disjoint arterials exist. In this case it is unlikely that a common cycle time will allow green waves to exist for both arterials eg. when the intersections of one arterial are more tightly spaced than in the other.

(Even though the arterials cannot be considered as a single arterial, they are not completely disjoint due to the effects of chaos theory. Otherwise it would make no sense to control the intersections in the area as  a single network.)

Another feature of considering very large networks is the possibility of traffic redirection. If it is detected - or predicted - that an arterial is, or will be, congested under current flow conditions it is sensible to redirect some traffic onto alternative routes. 
Redirecting traffic using traffic signals is a subtle technique which has not received much research yet though it could pose much more efficient than current traffic information systems, which inform road users of congestions and alternative routes, but allows them to ignore the advice.

In this section some in-depth discussions are given for three adaptive systems, which do not rely on offline optimization. The systems are:

\begin{enumerate}
\item \textbf{RHODES} by Pitu Mirchandani and Larry Head presented in \cite{44} - a hierarchical system for network-wide optimization
\item A \textbf{Phase-by-Phase} optimization strategy by Michael Shenoda and Randy Machemehl presented in \cite{1} - a system using the metaheuristic tabu search for determining greens for isolated intersections in a phase-by-phase manner
\item \textbf{DOGS} by danish Technical Traffic Solution (TTS) evaluated in the danish article \cite{dogs}, which provides criteria-based capacity increases along an arterial
\end{enumerate}

Comparisons will be made for the systems to highlight differences in areas such as \textit{detection}, \textit{prediction}, \textit{optimization strategy} and \textit{level of responsiveness}.

\subsection{RHODES}
RHODES approach to traffic signal optimization is a hierachical one with 3 layers of detail, see figure \ref{fig:rhodes_hierarchi}. They can be rougly be summarized accordingly:

\begin{enumerate}
\item The macroscopic layer performs \textit{dynamic network loading}, which involves observing changes in the aggregated flow data of the entire network due to variations in the OD matrices. This layer supplies estimates of link flows to the middle level in rough numbers eg. vehicles per hour.
\item The mesoscopic middle layer considers sectors of the network eg. an arterial. This \textit{network flow control} layer work in the detail level of platoons and average speeds. Green time is allocated to phases to accomodate the movements of the platoons and so coordination of signals is done at this level.
\item At the lowest level is \textit{intersection control} where vehicles are handled individually (a microscopic layer). Here the green times and phase ordering suggested by the middle layer are fine tuned.
\end{enumerate}

\begin{figure}[!ht]
\begin{center}
\includegraphics[scale=0.5]{rhodes_hierachy.png} 
\end{center}
\caption{The three levels of detail: network, sector, and intersection}
\label{fig:rhodes_hierarchi}
\end{figure}

An adaptive traffic control system must operate quickly in order to adapt signals to traffic in real-time. The RHODES platform has good decomposition opportunities and is pluggable ie. the upper level is a black box feeding the lower level with predictions and optimizations. At the time the article was written, the top level of RHODES had not received much development and thus only the middle and lower level are described herein.

\subsubsection*{Detection}
Detection methods are not discussed in detail in the paper. They can be of any technology including induction loops and video. 

Each link facing an intersection under intersection control are fitted with detectors to give the prediction methods optimum operating conditions. Detectors are placed in the beginning of the link so that the prediction method can use a longer horizon.

\begin{figure}[!ht]
\begin{center}
\includegraphics[scale=0.5]{rhodes_prediction-strategy.png} 
\end{center}
\caption{Detector placement and propogation of information in a simple grid}
\label{fig:rhodes_predict}
\end{figure}

\subsubsection*{Prediction}
The PREDICT method by the co-author, Head, is used to make predictions for individual vehicles. PREDICT is build for network prediction and relies on the fact that incoming flow to an intersection originates from adjacent intersections. This concept can be explained from figure \ref{fig:rhodes_predict} where traffic detected at $d_a$ is the sum of right-turning traffic at detector $d_r$, left turning traffic at $d_l$ and through-going traffic at $d_t$.

Thus given flow estimates for the links facing intersection B and turning probabilities for each link an estimate can be given for the inflow to intersection A from east. On the link between the two intersections there will be traffic entering and exiting the system, but these contributions - and losses - to the traffic, which can be measured at $d_a$ are expected to be very small.

Prediction of arrival times of the vehicles which has passed detectors $d_{\lbrace r,t,l \rbrace}$ depend on the current phase at intersection B and queue conditions. Mirchandani and Head has identified four cases, which cover arrivals to an intersection, which are summarized in table \ref{tbl:delaycases}.

\begin{table}[!ht]
\begin{center}
\begin{tabular}{l|ll}
 & \textbf{Green} & \textbf{Red} \\ \hline
\textbf{No queue} & 0 & $T_G$ \\
\textbf{Queue} & $T_Q$ & $T_G$ + $T_Q$
\end{tabular}
\end{center}
\caption{Delay incurred for a vehicle arriving to an intersection in various states where $T_Q$ is the time for ahead queue to clear and $T_G$ is the time to the next green.}
\label{tbl:delaycases}
\end{table}

In the cases involving queue there is, of course, a possibility that the vehicle will not be able to cross the intersection before several green phases have occured. This is likely to happen under high congestion when intersections are placed closely.

At the mesocopic level, network flow control, the APRES-NET prediction method is used. It is based on simulation and has similarities to PREDICT, though it works in the detail level of platoons and encompasses several intersections, not just the upstream ones but also those upstream of the upstream intersections and so on. Since the 2nd level must deliver complete suggestions for timing plans for each intersection (to be fine tuned by intersection control) it must run quickly. Performance is dependent on the number of intersections in the monitored sector and sector sizes can thus be scaled to match the speed requirements depending on hardware.

The prediction horizon for network flow control is 200-300 seconds. Cycle times for simple intersections with just a couple of phases vary between 60 and 150 seconds so this horizon is plenty to predict and respond to most types of fluctuations by performing phase skipping, phase reordering and assignment of phase durations. The intersection level control operates with a prediction horizon of 20-40 seconds and thus can only make make decisions on whether to lengthen or decrease the green time of phases within that horizon.

\subsubsection*{Optimization}
As in the dynamic model of section \ref{dynamicmodel} timing plans are described by phase ordering and duration independent of cycle time, splits and offsets. 
Optimization is performed on each level using prediction results for that level.

At the network flow control level the REALBAND algorithm forms progression bands (ie. \textit{green waves}) for platoons traversing the sector based on the predictions from APRES-NET. This is done by finding \textit{conflicts} between platoons, which will request access for conflicting phases at the same time. In this way a conflict can be regarded as the denial of green to a platoon due to the passage of another platoon. A decision tree within the optimization horizon of 200-300 seconds is build and explored to find the configuration with the fewest conflicts. This results in a set of phase orderings and green times for each intersection.

At the lower level a dynamic programming approach, COP, takes the results from REALBAND and distributes green time for some horizon, T, to the phases received from the above level. The phases and their ordering must be respected so as to not introduce conflicts, which have been resolved by REALBAND. For the same reason there are restrictions for the maximum change in either direction of the given green times, but COP is allowed to use its more detailed predictions to perform the mentioned fine-tuning of green times.

\subsubsection*{Evaluation}
RHODES has been implemented in software and evaluated in CORSIM as part of the evaluation for Federal Highway Administration (FHWA) inclusion in RT-TRACS. RT-TRACS is an effort to choose and standardize a peak performance traffic signal optimization system for the american traffic networks.

The simulation is done for an arterial of 9 intersections with steady increase and then decrease of traffic over a 2 hour period. This is a FHWA test case and the baseline traffic control system is semi-actuated control based on the results of offline tools including TRANSYT and PASSER, which represents the best-can-do from an offline approach and can be considered as a hard competitor. 

Testing shows that RHODES is more capable of exploiting the capacity of the arterial. As long as there is no congestion the throughput will match the demand and in the comparison RHODES can simply take more load before experiencing congestion.

Real adaptive systems should excel in the case of low demand, since the overcapacity will then allow RHODES to, rougly said, cater for each vehicle. The effect, compared to the semi-actuated control, is convincing with 50\% reduction in delays for low demand and 30\% reduction for high demand. This effect is expected to disappear when demand reaches the capacity of the arterial in which case, for both systems in the comparison, only throughput can be improved by increasing the green time along the arterial and maintaining proper coordination.

The simulation was run multiple times and for both throughput and delay it is clear that RHODES is more consistent and offers less variability from run to run.

\subsection{Phase-by-Phase}
The phase-by-phase (PP) system was developed to overcome a number of deficiencies, which seemed widespread in adaptive systems:

\begin{itemize}
\item Fixed cycle length and/or fixed step for variation of cycle length
\item Utilization of aggregated demand data only
\item Fixed coordination of signals along an arterial og through a network
\end{itemize}

The proposed overall scheme to improve upon these issues are the isolated optimization of intersections and more fine-grained tracking of vehicles.

The optimization process has been made independent from determination of the network state ie. detection and prediction and as such some of these subjects are mostly discussions and proposals for improvements.

\subsubsection*{Detection}
PP relies on individual tracking of vehicles to obtain an arrival based model. For networks this can be realized only with video detection with eg. license plate recognition; traditional loop detections cannot yet identify individual vehicles.

The PP system currently works for isolated intersections and so detection loops are sufficient to estimate arrival times. For detector placement, the authors suggest using simulation.

\subsubsection*{Prediction}
In the proposed form the PP system uses a Poisson process to generate interarrival times.
Alternatives are some form of time-series analysis or a Poisson process with variable mean. The use of detections made upstream could also be used, such as it is in RHODES.

The performance of PP is highly dependent on the ability of the chosen prediction system to generate proper forecasts but, as will be seen in the test results, the potential benefits are great.

\subsubsection*{Optimization}
The optimization procedure of PP seeks to minimize the stopped delay using input from the prediction process. The most widely used measure of effectiveness is stopped delay (see eg. \cite{9}, \cite{38} and \cite{31}) but the authors show that there is also a linear relationship to total travel time.

The following notation is used in the paper:

\begin{table}[!ht]
\begin{center}
\begin{tabular}{ll}
\hline
$H$ & Horizon of optimization \\
$cs$ & Cycle start time \\
$ce = cs+H$ & Cycle end time \\
$i = 1,...,N$ & Approach indexes \\
$k = 1,...,M$ & Phase indexes \\
$\lambda_k$ &  Is a partitioning of $H$ into phases and $0 = \lambda_{0} <  \lambda_{k-1} \leq \lambda_k \leq 1$ \\
$\lambda_{k_i}$ & The time into $H$ before the phase $k$, involving approach $i$, ends \\
$j$ & Vehicle id for predicted vehicle arrival within the horizon  \\
$t_{ij}$ & The arrival time for vehicle $j$ on approach $i$
\\ \hline
\end{tabular}
\end{center}
\caption{Notation}
\end{table}

$H$ should be in the order of the desired cycle time and $\lambda_k$ give the green splits and thus we have a full plan for the signal. Stopped delay can be calculated from this plan and the predicted arrivals. In Figure \ref{fig:pp_delay} this idea is sketched.

\begin{figure}[!ht]
\begin{center}
\includegraphics[scale=0.5]{phase-by-phase_delay-model.png} 
\end{center}
\caption{Calculation of stopped delay in the Phase-by-Phase system for an intersection with 3 approaches and 1 exit (no turning movements).}
\label{fig:pp_delay}
\end{figure}

In equation \ref{eqn:pp_delay} the rules for stopped delay are extracted when vehicle $j$ arrives on approach $i$ at $t_{ij}$.

\begin{equation}
delay = 
\begin{cases}
cs + H \cdot \lambda_{k_i-1} - t_{ij} & when \; t_{ij} \leq cs + H \cdot \lambda_{k_i-1} \; (before\;green)  \\
cs + H - t_{ij} & when \; t_{ij} \geq cs + H \cdot \lambda_{k_i} \; (after\;green)  \\
0 & otherwise
\end{cases}
\label{eqn:pp_delay}
\end{equation}

In the second case of equation \ref{eqn:pp_delay} vehicles incur stopped delay since they arrive after their approach has been served green time in the planning horizon. Thus they will not be served before the next green, which has not yet been planned, and stopped delay is accumulated until then. This is called carryover since vehicles are carried over into the next cycle.

PP also takes into account queue startup delay and thus cover the most critical sources of delay. However PP makes the assumption that the granting of green time to a phase will cause the approaches to be cleared completely ie. no vehicles must experience more than one green phase before they can leave.

The objective function is defined using these delay terms and thus optimization can be done by making changes in the $\lambda_k$-values within some critical points in horizon. Looking at Figure \ref{fig:pp_delay} it is seen that approach 2 is served green time until $\lambda_1 H$, supressing green from approach 1 and 3, which both have arrivals. By switching phase immediately after $t_{21}$ (setting $\lambda_1 = (t_{21} - cs)/H$) approach 1 could receive green until immediately after $t_{12}$ and so on. This example involves switching of the phase order, which was turned off in the paper.

In the PP paper \cite{1} a solution method using the above scheme is presented as a combinatorial problem. But the number of combinations increase exponentially with the number of arrivals and the number of phases. Therefore a tabu search is employed. Websters formula for optimum green time splits is used in the Proportional Heuristic to obtain a good initial solution and a 1-bit neighborhood function is defined by making changes to a single value in $\lambda_k$ (a low influence move), preserving the phase order. Candidates for the step of these changes range from the transmission time of an electronic signal ($\approx 10^{-3}s$) to the minimum headway between two vehicles travelling in a platoon ($\approx 2s$ cf. Greenshields et al. 1947).
A high influence move, which reorders phases, is also described but is turned off, as mentioned.

\subsubsection*{Evaluation}
Shenoda and Machemel compares the results of their metaheuristic search to pretimed and actuated signal control settings obtained from the CORSIM microsimulator using the test intersection in Figure \ref{fig:pp_intersection}.

\begin{figure}[!ht]
\begin{center}
\includegraphics[scale=0.5]{phase-by-phase_testing-intersection.png} 
\end{center}
\caption{4-phase intersection from experiment \#2}
\label{fig:pp_intersection}
\end{figure}

The intersection was subjected to 8 different data sets of arrival times. In Figure \ref{fig:pp_improvements} the stopped delays for the PP system is compared to the simulation results using pretimed plans and standard traffic actuated control.

\begin{figure}[!ht]
\begin{center}
\includegraphics[scale=0.5]{phase-by-phase_improvement_ratios.png} 
\end{center}
\caption{Improvement factors of PP compared to pretimed and actuated control in the 4-phase intersection of Figure \ref{fig:pp_intersection}}
\label{fig:pp_improvements}
\end{figure}

The results should only be taken as indications since a number of assumptions were made for PP that the prediction algorithm supplied perfect information. In addition the actuated control was only semi-actuated since detectors were only used in one direction, the other being timed as in the pretimed case, which was optimized using Websters formulas.

In spite of these issues it is interesting to observe the the (semi-)actuated control strategy is not always superior to pretimed plans. It is clear, however, that both strategies are outperformed by PP. Under the given assumptions - in particular that concerning accuracy of predictions - PP can be used to establish a baseline for the best possible performance. This becomes even more true when the phase ordering constraint is dropped allowing reordering and skipping of phases.

Unfortunately Shenoda and Machemehl do not test PP on a network or even along an arterial. The optimization procedure does not consider coordination in itself though it is proposed that the prediction routine should consider departures from adjacent intersections, such as the method by Head employed in RHODES. It is speculated that such propagation of departure information could give rise to some coordination, depending on the horizon of optimization.

\subsection{DOGS}
DOGS is an extension of the DOG system. The first 3 letters can be directly translated from danish to \textit{dynamic optimization of greens} and the appended S of the herein regarded system means \textit{coordination}. Thus DOG is an traffic actuated optimizer for single intersections, as described in section \ref{actuated}, and DOGS add coordination.

DOGS is a criteria-based system which relies on a common cycle time for coordination. The intended area of application is traffic signals along arterials, which see a high fluctuation of demand. 

The Danish Road Directorate (DRD) has implemented DOGS along several arterials in Denmark. 
Among these are the Glostrup and Herlev implementations along the O3 ring-road. They see high increases in demand in the morning and afternoons when commuters come to and leave work places. Normally such fluctuations can be handled well by a fixed number of timing plans (see section \ref{pretimed}), which are changed according to the time of day, but the area also see more unpredictable types of fluctuations when accidents occur on the nearby highway, M3, or when M3 lanes are closed due to expansions. For these reasons DRD decided implement a system which could handle such situations.

The purpose of DOGS is to increase the capacity of the arterial in these cases - in low traffic situations pretimed plans will be used. The capacity increase comes by adjusting a common cycle time and thus allocating more green time to the major phases. This will cause increased delays for the minor roads, but may prevent queues from reaching the previous intersection - or even prevent queues in cases of light congestion.
DOGS is also capable of providing priority to buses by extending the green time when buses are near an intersection.

At present the system must be tailored to the environment in which is must operate. For this reason the following sections will use the Herlev area as a reference area in order to explain certain concepts.

\subsubsection{Detection}
Loops are placed in the immediate downstream of signal controllers and connected to closest controller to reduce the wire length. 

The criteria which DOGS use are based on the load on the detectors and the number of vehicles detected per cycle. Furthermore the detectors are capable of measuring vehicle speeds, though this is not taken into consideration in any of the criteria.

\subsubsection{Prediction}
DOGS is a purely traffic actuated system and no prediction is used when the system is activated due to heavy traffic conditions.

In spite of the flexibility of the system this is a point which puts high demand on the implementing traffic engineers since traffic through the arterial must be assessed manually when the system is put into production as well as during maintenance.

\subsubsection{Optimization}
Since DOGS only kicks in under congested or near-congested conditions (for the Herlev area when the load exceeds 60\%) it is simple to optimize the throughput since, as explained earlier, any increase in green time will just allow more vehicles to pass (the phase is never emptied from vehicles).

The load measurement in Herlev is extracted from detectors, which are strategically placed in the ends of the arterial. During an interview with DRD and TTS it was explained that the traffic from minor roads can be neglected in the load measurement, though both parties would be interested in seeing the effect of increasing the number of detectors used.

The common cycle time is set according to the load level and to avoid sudden, major changes in cycle time and temporary loss of coordination, cycle time and green times are adjusted to within some maximum per cycle. 

Coordination is achieved by running the signals on the common cycle time, but offsets are not adjusted when the common cycle time changes so this issue should receive further investigation.

\subsection{Comparison}
