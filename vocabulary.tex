\label{vocabulary}
There are many terms concerning traffic signal optimization. This section attempts to extract the most important terms and give solid descriptions.


\begin{description}
	\item[MOE] Measure Of Effectiveness. Also referred to as the performance index (PI).
	Most often used is the average delay, also common is the time in (traffic) network and number of stops. 
			
	\item[Cycle] The turnaround time for all phases of a traffic signal to complete.
	
	\item[Phase] A phase of a traffic signal corresponds to a particular state. 
	For instance there may be a green phase in the north-south direction for a two-way intersection (which implies red phase in the east-west direction).
	More complex intersections may also split the green phase up into a left-turn phase and a right-and-through phase (for left-side driven vehicles).
	
	\item[Split] (Or \textit{cycle}-split) is the green-to-red time ratio split. Usually a higher ratio means more green time.
	
	\item[Traffic assignment] also known as equilibrium flow, is the determination of vehicular flow along origin-destination paths and, consequently, along links in a traffic network. The information gained from a correct traffic assignment is vital to the determination of traffic signal settings and vice-verse. Thus many articles consider the two problems simultaneously.

\item[Artery] is a main-path through a traffic network. It will generally face higher demand than auxilliary paths.

\item[Traffic network] is most often described mathematically as a graph $G(V,E)$ where $V$ is the set of intersections controlled by a traffic signal and $E$ is the set of roads connecting the intersection. A path is thus a route through the network crossing a least one signalized intersection.

	\item[A platoon] is group of vehicles travelling together. A platoon can be detected by observing the time between on vehicle and the next and applying the critical headway threshold, see \cite[sct. 2]{25}. 
Platoons are formed both as a consequence of car-following behaviour, which is used in simulation frameworks such as \cite{treiber-2000-62}, but also due to the batch-like nature which is imposed on the traffic by traffic signals.
\end{description}
