\label{sec:dogs}

DOGS is an extension of the DOG system. The first 3 letters can be
directly translated from Danish to \textit{dynamic optimization of
greens} and the appended S means
\textit{coordination}. Thus DOG is a traffic-actuated optimizer for
single intersections, as described in section \ref{insec:actuated},
and DOGS adds coordination. The DRD has implemented DOGS along several
arterials in Denmark.

DOGS is a criteria-based system which relies on a common cycle time
for coordination. The intended area of application is traffic signals
along arterials, which see a high fluctuation in demand.

The purpose of DOGS is to increase the capacity of the arterial in
high demand periods and revert to offline-optimized, pretimed plans in
low traffic situations. The capacity increase is realized by
increasing the common cycle time and allocating the extra green time per
cycle to the phases along the arterial. This will cause increased
delays for the minor roads, but may prevent queues from reaching the
previous intersection, or even prevent queues in cases of light
congestion.  DOGS is also capable of providing priority to buses by
extending the green time when buses are near an intersection.

At present the system must be tailored to the environment in which it
operates. For this reason the following sections will use the Herlev
area in Denmark as a reference area in order to explain certain
concepts. Figure \ref{fig:dogs_herlev} shows the layout of this
network.

\begin{center}
{\bf Figure 6 in here}
\end{center}

\subsubsection*{Prediction}

DOGS is a purely traffic-actuated system and no prediction is used
when the system is activated due to heavy traffic conditions.  In
spite of the intended flexibility of the system this is a point which
puts high demand on the implementing traffic engineers since traffic
through the arterial must be assessed manually when the system is put
into production as well as during maintenance.

An alleviating point to the lack of prediction is the fact that the
current arterials under DOGS control are relatively small, and static
predictions can be made by an experienced traffic
engineer. Furthermore, since DOGS only operates under high load
conditions, predictions become less valuable - or superfluous, even -
because all that can be said about the arterial in this case is that
it is heavily loaded with traffic.

\subsubsection*{Optimization}

Since DOGS only kicks in under congested or near-congested conditions
(for the Herlev area when the load exceeds 60\%) it is simple to
optimize the throughput since any increase in green time will just
allow more vehicles to pass (the phase is never emptied of vehicles).

That DOGS only operates during high-congestion levels is an unusual
trait for an adaptive system since they usually excel in optimization
under \textit{normal} ie. uncongested load conditions (see the
comparison of RHODES and a semi-actuated system in section
\ref{sec:rhodes}). This can be explained by the lack of an
explicitly defined objective function and optimization routine.

The objective is to keep the load degree (load/capacity) for the most
heavily loaded intersection below 90\%.  To do this the common cycle
time and green times are set according to the load level. The
adjustments are made with a few seconds per cycle to avoid sudden,
major changes in cycle time and temporary loss of coordination.

Coordination is achieved by running the signals on a common cycle
time, but offsets are not adjusted when the common cycle time changes,
so this issue should receive further investigation.

A set of non-overlapping criteria are used to select a program with
the appropriate capacity for the detected inflow. For a technical
description of these criteria cf. \citet{forprojekt}

%% When naming the inflow detectors in the north- and south ends $DN$ and $DS$, respectively, the transition from pretimed control to adaptive control is decided by the constraint:
%% \begin{eqnarray*}
%% Intensity(DN) > I_{enable} & \vee & Intensity(DS) > I_{enable} \\
%% & or & \\
%% Load(DN) > L_{enable} & \vee & Load(DS) > L_{enable}
%% \end{eqnarray*}

%% For switching back to pretimed control this constraint must hold:
%% \begin{eqnarray*}
%% Intensity(DN) < I_{disable} & \wedge & Intensity(DS) < I_{disable} \\
%% & and & \\
%% Load(DN) < L_{disable} & \wedge & Load(DS) < L_{disable}
%% \end{eqnarray*}

%% To avoid hysteresis ie. constant enabling and disabling of dynamic control:
%% \begin{eqnarray}
%% I_{enable} - I_{disable} & \geq & I_{\varepsilon} \label{eqn:hysteresis_intensity} \\ 
%% L_{enable} - L_{disable} & \geq & L_{\varepsilon} \label{eqn:hysteresis_load} \\
%% I_{\varepsilon},L_{\varepsilon} & > & 0 \label{eqn:hysteresis_limits}
%% \label{eqn:hysteresis}
%% \end{eqnarray}

%% DOGS exhibits a dynamic behaviour because the permitted cycle time extensions are divided into \textit{programs} according to the intensity and load levels in the ends of the artery. These program constraints take a form which is similar to the enable- and disable constraints.

%% When deciding whether to remain in the current program, $i-1$, or switch to a program for higher demand, $i$, the following relation must be satisfied:

%% \begin{eqnarray*}
%% I_{enable,i+1} \geq & \max(Intensity(DN),Intensity(DS)) & > I_{enable,i} \\
%% & \vee & \\
%% L_{enable,i+1} \geq & \max(Load(DN),Load(DS))  & > L_{enable,i}
%% \end{eqnarray*}

%% The decision of switching from program $i$ to the program for lower demand, $i-1$, is determined by this relation:

%% \begin{eqnarray*}
%% I_{disable,i-1} \leq & \max(Intensity(DN),Intensity(DS)) & < I_{disable,i} \\
%% & \wedge & \\
%% L_{disable,i-1} \leq & \max(Load(DN),Load(DS))  & < L_{disable,i}
%% \end{eqnarray*}

%% In the DOGS controlled area in Herlev there are 8 different programs to choose from and the enable and disable thresholds for programs respect the ordering implicit in the above equations:

%% $$\lbrace I,L \rbrace_{enable,i+1} > \lbrace I,L \rbrace_{enable,i}$$
%% $$\lbrace I,L \rbrace_{disable,i-1} < \lbrace I,L \rbrace_{disable,i}$$

%% To avoid hysteresis between programs the same constraints (equations \ref{eqn:hysteresis_intensity}-\ref{eqn:hysteresis_limits}) as for switching between pretimed and dynamic control applies.

\subsubsection*{Evaluation}

Tests have shown that the system is indeed capable of increasing the
capacity, with reduced queue lengths as a result. When the arterial in
Herlev is at or above moderate load, DOGS will increase the capacity by
15-25\% compared to the capacity if only the pretimed plans were in
use.
