\section{Introduction}

Road traffic is an essential part of modern society and has put a high
demand on road networks. Figures from a recent study from the Danish
ministry of transport \cite{47} shows traffic has increased by 50\%
since the 80's and that cars and busses are responsible for more than
90\% of all person transport in Denmark (totalling 68 milliom
kilometers).

Traffic network congestion causes delays which adds substantial costs
to the society and businesses on a daily basis and also increases
emissions and the risk of accidents. In the beforementioned study it
is reported that in 2002 people where spending 100,000 hours in total
in queue in the Greather Copenhagen road infrastructure, this
corresponds to an economic loss of more than 750 million Euros.  

To alleviate congestion, public transport can be improved or the
infrastructure can be expanded. In urban areas, the latter is often
impossible due to residential areas adjacent to the existing roads.
A more subtle way to improve the network performance is to make better
use of the existing roads, which can be done by proper setting of
traffic signal parameters.

It is estimated that the proper use of intelligent traffic systems
including intelligent traffic signals can increase the capacity of the
road network in the Greater Copenhagen area by 5 to 10\%.

Traffic signals are employed in urban areas in order to control
traffic flows and avoid collisions by fairly distributing the
right-of-way for adjoining road segments and to ensure a smooth flow
of traffic. If not set appropriately traffic signals may seriously
affect the traffic flow through the road network and so great care
should be taken when choosing the parameters for an intersection.

The literature has many suggestions for the intelligent setting of
traffic signals, ranging from purely statistically based methods
developed in the early 60's over actuated traffic signals to highly
adaptive and cooperative methods, which can be realized using actual
flow information supplied by traffic detectors. This paper gives a
survey of the literature with special emphasis on adaptive methods,
which attempts to coordinate traffic signals in larger networks so as
to optimize some network-wide performance index such as number of
stops or delay.

The paper is structured as follows. XXX Oversigt skal skrive naar
resten er paa plads XXX. Finally in section \ref{sec:conclusion} we
summarize the findings and come with recommendations for future works.
