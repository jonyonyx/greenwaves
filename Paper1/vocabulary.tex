\label{sec:vocabulary}

Fortunately, the terms of traffic signal optimization seem to be
fairly standardized and most articles share terminology.

The traffic network is modelled as a directed graph $G(V,E)$ where $V$
is the set of intersections controlled by a traffic signal and $E$ is
the set of roads connecting the intersections. A path is thus a route
through the network.

In the context of adaptive traffic control an {\em artery} defines a
main-path, the major road, through a traffic network. Such an artery will
generally face higher demand than minor roads adjoining the
artery. The time it takes for a traffic signal to get from the start
of the green light through the yellow and red and until it again
becomes green is denoted the {\em cycle time}. The cycle time is one
of the most important variables in setting up the traffic signal.

A phase (or stage) corresponds to a particular state of the red and
green lights of the traffic lights in an intersection. For instance
there may be a green phase in the north and south direction for a
two-way intersection (which implies red phase in the east-west
direction). When \textit{a phase} is mentioned, it is uduslly implicit
that it is the green phase.

The performance is evaluated by considering the {\em Measure Of
Effectiveness} (MOE). Most common is the average delay, but the travel
time through the network and number of stops, or some combination of
these, is also common.

%% \item[Interphase green] Also known as lost time, is a small amount
%% of time inserted as a buffer between two phases. During the lost
%% time the lights can be either red in all directions or, as in
%% Denmark, amber lights can be used to introduce a buffer. The
%% purpose of the buffer is to allow vehicles, which entered during
%% the last phase, to exit the intersection before it is flooded by
%% vehicles from the next phase.

%% \item[Queue spillback] This phenomenon occurs when a queue reaches from a downstream intersection to the preceding intersection, effectively preventing traffic from leaving the upstream intersection.

%% \item[Saturation flow rate] The number of vehicles that can flow on a link at the maximum allowed speed (free-flow) during a period of time.

%% 	\item[Cycle-split]  For instance in a four-phase intersection with cycle time of $100s$ and cycle splits $\left( 0.3, 0.2, 0.3, 0.2 \right)$ the actual green times will be $\left( 30s, 20s, 30s, 20s \right)$.	

%% \item[Time horizon] The amount of time, which is taken into consideration while optimizing signal settings or making predictions. Since predictions of the future traffic becomes more and more fuzzy the deeper one looks a paradox arises: using a short time horizon the optimizations might prove to be flawed when it fails to see a clever decision, but with longer time horizons the predictions themselves become flawed and may mislead the optimization.
