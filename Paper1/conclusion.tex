\label{conclusion}
This survey has covered the fundamental areas of traffic control by signals including definition of mathematical objectives and models for traffic networks (sections \ref{scope} and \ref{theproblem}) and signal types (section \ref{signal_types}). 

Simulation as a tool for evaluation of solution quality is discussed in section \ref{evaluation}. It is clear that microsimulation is preferred, but performs slower than the meso- or macroscopic solutions, which rely on deterministic principles such as the SUE formed on the basis on Wardrop's results in section \ref{usereq}. Webster (section \ref{webster}) used Wardrops result in a queuing model to give an approximate formula for delays at intersections, which has been widely used to find cycle- and green times under stochastic equilibrium assumption.

The preference for microsimulation supports the notion that traffic is of dynamic nature and demand may not always fit into the aggregated view from OD matrices. The emphasis of the survey has been on adaptive systems, which can accommodate for the fluctuations in traffic. 
The flexibility of the model (or lack of) underlying the optimization is a determining factor to the level of adaptiveness that is achievable. The offline optimization systems (section \ref{offline}) all operate with the common cycle time concept, which allows coordination to be set up by offsetting downstream intersections. The common cycle time is prohibitive when adaptive systems try to react to (predicted) arrival of single vehicles or platoons of vehicles, even.

There exists several models for traffic networks, which are not based on the periodic behaviour of offline systems to perform coordination. Instead they assign green time to phases in some order, which is optimal given the detected and predicted traffic. 
In section \ref{adaptive_cooperation} three systems were discussed in detail. They were selected to give examples of adaptive control at different levels (network, arterial and intersection). Two of them (network and intersection control) use the direct assignment / phase assignment methodology and the last (DOGS for arterial control) operate with a classical model, but is designed so that the mentioned problems are neglible.

DOGS is an example of a simple, but efficient, solution to dynamic capacity adjustment for an arterial. It has been implemented for multiple highway arterials and has thus proven that it can supply capacity increases when needed as well as readjust priority to minor roads once the traffic flow on the arterial diminish.
DOGS has little mathematical background, however, and has not been simulated prior to implementation. 
In a future project it would be interesting to introduce a mathematical foundation for DOGS, preferably on the basis on some established arterial progression scheme such as REALBAND. Before and after scenarios could be simulated to determine, what improvements are possible by going from a system adjusted by ad-hoc methods to a truly optimized system.