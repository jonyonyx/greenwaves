\label{sec:conclusion}

The emphasis of this survey has been on adaptive systems, which can
accommodate for the fluctuations in traffic.  The flexibility of the
model (or lack of it) underlying the optimization is a determining factor
regarding the level of adaptiveness that is achievable. The offline
optimization systems (section \ref{sec:offline}) all operate with the
common cycle time concept, which allows coordination to be set up by
offsetting downstream intersections. The common cycle time is
prohibitive when adaptive systems try to react to (predicted) arrival
of single vehicles or platoons of vehicles, even.

There are several models for traffic networks, which are not based
on the periodic behaviour of offline systems to perform
coordination. Instead they assign green time to phases in some order,
which is optimal given the detected and predicted traffic.  In section
\ref{sec:adaptive_cooperation} three systems were discussed in
detail. They were selected to give examples of adaptive control at
different levels (network, arterial and intersection). Two of them
(network and intersection control) use the direct assignment / phase
assignment methodology and the last (DOGS for arterial control)
operates with a classical model, but is designed so that the mentioned
problems are negligible.

DOGS is an example of a simple, but efficient, solution to dynamic
capacity adjustment for an arterial. It has been implemented for
multiple highway arterials and has thus proven that it can supply
capacity increases when needed as well as adjust priority to minor
roads once the traffic flow on the arterial diminishes.  DOGS has little
mathematical background, however, and has not been simulated prior to
implementation.  In a future project it would be interesting to
introduce a mathematical foundation for DOGS, preferably on the basis
on some established arterial progression scheme such as
REALBAND. Before and after scenarios could be simulated to determine,
what improvements are possible by going from a system adjusted by
ad-hoc methods to a truly optimized system.
